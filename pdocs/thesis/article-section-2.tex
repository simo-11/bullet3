\section{Adding plasticity to physics engine}

In this section some central terms are explained and main differences between traditional structural 
analysis and physics engines are reviewed.

Velocity-based formulation is not typically used in structural or mechanical engineering.
 \citet[p.~45]{erleben.thesis} provides reasoning and theoretical details why it is so popular in 
constraint-based rigid body simulation. 
Main reason is that collision handling can be done without additional procedures.
In structural analysis solution method is selected carefully based on needed features.
For most complex scenarios finite element method is used.
In most cases static small displacement solution using displacement based boundary conditions is used.
For large displacement static analysis analysis loading is applied in substeps and 
displacements are used to update element mesh.
Further enhancements are material nonlinearity and dynamic analysis.
Physics engine provides dynamic analysis with large displacements.

Material plasticity has typically taken into account in games by using suitable coefficient of restitution.
This provides reasonable means to simulate loss of energy in collisions.
Simulation of breaking of objects made of ductile material can be made more realistic by splitting rigid objects
to multiple parts which are connected by energy absorbing joints.
As stress-strain curve may not be familiar to all readers some basics are provided here.
Typical engineering stress-strain curve of ductile steel is shown in \ref{fig:sscurve}.
E.g. \citet{dowling} provides detailed descriptions of engineering and true stress-strain curves.
Stress-strain curve is not drawn to scale as elastic strain could not be seen as it is typically 0.001 to 0.005.

In this work elastic-fully plastic material model is used in most scenarios.
It allows realistic simulations for most scenarios.
Having elastic part allows elastic displacements for slender structures. 
Elastic part is ignored in method suggested in this work if deformation is related
to higher frequency than integration stability would allow.
Strain hardening part of stress-straing curve could probably be taken into account but it has not been tried in this work.
It should be noted that geometry
of objects is not updated during analysis and thus engineering stress-strain properties should
be used even with strain hardening.

Strain hardening is taken into account in this work mainly by assuming that plasticity in bending expands, \citet[p.~672]{dowling}.
Material that starts to yield first is hardened and yielding moves slightly.
This can be seen e.g. by bending paperclip. It does not break at low angles but can take few full bends. 

\begin{figure}[htb!]
\centering
\begin{tikzpicture}
\coordinate (Y) at (0.2,4);
\draw[->] (0,0) -- (10,0) node[right] {\large{$\epsilon$}};
\draw[->] (0,0) -- (0,6) node[above] {\large{$\sigma$}};
\draw(0,0) -- (Y) -- (2,4) .. controls (7,6) .. (10,5);
\draw[dashed](0,4) -- (Y);
\node at (-0.2,4) [align=right] {$f_y$};
\end{tikzpicture}
\caption{Engineering stress-strain curve of ductile steel (not to scale).}
\label{fig:sscurve}
\end{figure}

Difference between elastic and plastic section modulus is shown in \ref{fig:wp}. 
If stress is below yield limit, stress and strain are linear within cross section.
If cross section is fully plastic, stress is assumed to be at yield level over whole cross section and 
so plastic section modulus is higher than elastic section modulus.

\begin{figure}[htb!]
\centering
\begin{tikzpicture}
\coordinate (S) at (2.5,5);
\draw (0,5) -- (4,5) ;
\draw (0,0) -- (4,0) ;
\draw (2,0) -- (2,5) ;
\draw (1.5,0) -- (S); 
\node[above] at (S) [align=center] {\large{$\sigma<f_y$}};
\end{tikzpicture}
\hspace{1cm}
\begin{tikzpicture}
\coordinate (S) at (3,5);
\draw (0,5) -- (4,5) ;
\draw (0,0) -- (4,0) ;
\draw (2,0) -- (2,5) ;
\draw (1,0) -- (1,2.5) -- (3,2.5) -- (S); 
\node[above] at (S) [align=center] {\large{$\sigma=f_y$}};
\end{tikzpicture}
\caption{Stress distribution under elastic and plastic loads.}
\label{fig:wp}
\end{figure}

Basic idea in this work can be tested with any framework having motors and hinge constraints.
This can be done by setting target velocity of motor to zero and limiting maximum motor impulse to plastic moment 
multiplied by timestep.

Further enhancements were created and tested by forking \bullet\ source code
and adding new constraints \cite{pbullet}.
Constraint processing in \bullet\ is based on ODE, \cite{ode}.
Mathematical background and detailed examples are available by \cite{ode.joints}.
Equations \ref{eq:constraintEquation}, \ref{eq:lambdaLow} and
\ref{eq:lambdaHigh} 
are created for each constraint.

\begin{equation} \label{eq:constraintEquation}
J_1 v_1 + \Omega_1 \omega_1 + J_2 v_2 + \Omega_2 \omega_2 = c + C \lambda
\end{equation}

\begin{equation} \label{eq:lambdaLow}
\lambda \geq l
\end{equation}

\begin{equation} \label{eq:lambdaHigh}
\lambda \leq h
\end{equation}

Main parameters  and corresponding fields in \bullet\  
 are described in table \ref{tab:constraintParameters}.

\begin {table}[htb!]
\begin{center}
\begin{tabular}{|c| l| l|}
\hline
{\bf Parameter} & {\bf Description} & {\bf btConstraintInfo2 pointer}\\  \hline
$J_1, \Omega_1$ & Jacobian & m\_J1linearAxis, m\_J1angularAxis \\ 
$J_2, \Omega_2$ & & m\_J2linearAxis, m\_J2angularAxis \\ \hline
$v$ & linear velocity & \\ \hline
$\omega$ & angular velocity & \\ \hline
$c$        &  right side vector   & m\_constraintError \\ \hline
$C$  & constraint force mixing & cfm \\  \hline
$\lambda$ & constraint force &  \\ \hline
$l$ & low limit for constraint force & m\_lowerLimit \\ \hline
$h$ & high limit for constraint force & m\_upperLimit \\ \hline
\end {tabular}
\end{center}
\caption {Constraint parameters} \label{tab:constraintParameters} 
\end {table}
