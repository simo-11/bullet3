% !TeX root = article.tex
\begin{abstract}
We introduce efficient method to simulate ductile fracture in existing physics engines.
\end{abstract}

\section{Introduction}

Computational techniques used in modern computer games are able to describe
complex dynamic systems such as those found in military and other motorized vehicles
with a high level of accuracy while still being able to solve the equations of motion in real time.
Film production and the computer game genre of war games
are two areas that have greatly benefited from rapidly improving simulation technology. 
In practice, software for such games is often written using open-source physics engine platforms  
like ODE - \cite{ode}, \bullet\ - \cite{bullet} and Box2D - \cite{box2d}.

Computational methods used in physics engines are divided into modules that handle collision detection and 
contact description and modules that handle solution of equations of motion in real time. 
The equations of motion that need to be 
solved can be subdivided into equations related to motion, constraints and collisions. 
Velocity-based formulation is typically used in constraint based rigid body simulation, \cite{erleben.thesis}. 
Friction is typically taken into account and mechanical joints are handled by constraint equations,
\cite{erleben.thesis}.

Plasticity is not typically taken into account in gaming solutions and 
destruction of various bodies often takes place based on collision or impulse exceeding predefined limit.
Nevertheless, breaking of steel or reinforced concrete structures using this approach 
is not appropriate if the simulation is to look realistic. Theory for handling of plasticity 
has been presented already in \cite{cg1988}. \cite{muller2005meshless} 
present a method for modeling and animating of elastic and plastic bodies in real time using 
point based animation. This approach has not been widely used in computer games.  
One major issue is collision handling of deformable bodies.

This study will introduce an approach to account for plastic deformation in game applications.   
In the introduced method, plastic deformation takes place if the force or moment exceeds a predefined 
limit, deformation absorbs energy and joint breaks if plastic capacity is exceeded. 
The approach is based on using joint motors to model plasticity. 
The study extends a method introduced by
\citet{erleben.thesis} 
%\citet[p.~90]{erleben.thesis} 
which was originally proposed for modelling friction in joints. 
In the introduced method adjacent bodies are connected by motors. 
Motor power production limits are estimated based on the plastic section modulus. 
Joint breaking is based on summing plastic deformation and comparing it to a
predefined material based limit. The elastic part of deformation is modelled by employing 
a spring based on modification of an existing constraint in \bullet.

The approach presented in this work can be used in the gaming industry to provide more realistic 
simulations without significant extra work. For gaming purposes, the presented method works 
best in scenarios where the connected parts are heavy. This allows a normal 
integration timestep to be used without stability issues. 
This methodological approach enbles established structural analysis
methods to be combined with modern simulation frameworks.

