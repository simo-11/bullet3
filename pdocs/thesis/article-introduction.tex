\section{Introduction}

Physics engine is computer software that provides an approximate simulation of physical system. 
E.g. ODE - \cite{ode}, \bullet - \cite{bullet}, Box2D - \cite{box2d} are open source physics engines that are  widely used in 
areas where fast solutions are required. Typical industries are games and film productions.

Computational methods used in physics engines are divided to modules that handle collision detection and modules that handle simulation.
Simulation can further be subdivided into time control, motion solver, constraint solver and collision solver modules.
Velocity-based formulation is typically used in constraint based rigid body simulation 
frameworks as collisions cannot be handled easily in acceleration based formulations. 
Friction is typically taken into account. Joints are handled by constraint equations.
Detailed description of various components  can be found in e.g. \cite{erleben.thesis}.

Plasticity is not typically taken into account in gaming solutions. 
Breaking of various objects typically takes place based on collision or impulse and breaking of steel or 
reinforced concrete structures does not look realistic.
Theory for handling of plasticity has been presented already in \cite{cg1988}. 
\cite{muller2004point} and \cite{muller2005meshless} present a method for 
modeling and animating of elastic and plastic objects in realtime using point based animation but it is not widely used.
On major issue is collision handling of deformable objects.

Goal of this work is to find method that can be used in widely used rigid body based physics engines to make simulation more realistic.
Main target is that plastic deformation takes place if force or moment exceeds given limit, deformation absorbs energy and 
joint breaks if plastic capacity is exceeced. Suggested method is based on using joint motors to model plasticity. 
\cite[p.~90]{erleben.thesis} suggests similar method for modelling friction in joints.
Adjacent objects are connected by motors. Motor limits are calculated based on plastic section modulus.
Joint breaking is simulated by summing plastic deformation and comparing it to predefined material based limit.
Elastic part of deformation is modelled by spring based simulation which is based on modification of existing constraint in \bullet.

Methods presented in this work can be used in gaming industry to provide more realistic simulations without significant extra work. 
For gaming purposes presented method works best in scenarios where connected parts are relatively heavy. 
This allows normal integration timestep to be used without stability issues.
This kind of metodology also opens quite large area of combining old structural analysis methods to modern simulation frameworks.
