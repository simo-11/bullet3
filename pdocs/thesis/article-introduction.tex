\section{Introduction}

The application of modern computer game techniques enables the description of complex dynamic systems 
such as military vehicles with a high level of detail while still solving the equations in real-time.
Film production and war games, in particular, is a key area that have benefited from simulation technology. 
In practice, games are often accomplished using an open-source platform such 
like ODE - \cite{ode}, \bullet\ - \cite{bullet} and Box2D - \cite{box2d}.

Computational methods used in physics engines are divided to modules that handle collision detection and 
contact description and modules that handle solution of equations in real-time. Equations need to be 
solved can further be subdivided to be associated to motion, constraints and collisions. 
Velocity-based formulation is typically used in constraint based rigid body simulation 
frameworks as collisions cannot be handled easily in acceleration based formulations. 
Friction is typically taken into account and mechanical joints are handled by constraint equations.
Detailed description of various components can be found in e.g. \cite{erleben.thesis}.

Plasticity is not typically taken into account in gaming solutions. 
Breaking of various objects typically takes place based on collision or impulse.
Nevertheless, breaking of steel or reinforced concrete structures using this approach 
is not appropriate making a simulation to look unrealistic. Theory for handling of plasticity 
has been presented already in \cite{cg1988}. \cite{muller2004point} and \cite{muller2005meshless} 
present a method for modeling and animating of elastic and plastic objects in real-time using 
point based animation. This approach is not been widely used in simulation applications.  
On major issue is collision handling of deformable objects.

This study will introduce an approach to account plastic deformation in game applications.   
In the introduced method, the plastic deformation takes place if force or moment exceeds given 
limit, deformation absorbs energy and joint breaks if plastic capacity is exceeced. 
The approach is based on using joint motors to model plasticity. \cite[p.~90]{erleben.thesis} 
suggests similar method for modelling friction in joints. Adjacent objects are connected by motors. 
Motor power production limits are estimated based on plastic section modulus. 
Joint breaking is accounted by summing plastic deformation and comparing it to 
predefined material based limit. Elastic part of deformation is modelled by employing 
spring description which is based on modification of existing constraint in \bullet.

Approach presented in this work can be used in gaming industry to provide more realistic 
simulations without significant extra work. For gaming purposes presented method works 
best in scenarios where connected parts are relatively heavy. This allows normal 
integration timestep to be used without stability issues. 
This kind of metodology also opens large area of combining old structural analysis
methods to modern simulation frameworks.

