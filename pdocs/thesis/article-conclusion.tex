\section{Conclusion}

This study has presented an approach to account plastic deformation in 
velocity based formulation.
In the introduced method, the plastic deformation takes place if force or moment exceeds given 
limit, deformation absorbs energy and joint breaks if plastic capacity is exceeded. 
Approach presented in this work can be used in gaming industry to provide more realistic 
simulations without significant extra work. 
For gaming purposes presented method works 
best in scenarios where connected parts are relatively heavy. This allows normal 
integration timestep to be used without stability issues. 

This kind of metodology also opens large area of combining  structural analysis
methods to realtime simulation frameworks.
Handling of significant axial and shear forces and moments at same time
is one possible area of further studies.
Handling of more complex structures is already being studied by author.

There is also work to be done in the area of performance optimization.
One possible area is reusing once calculated values if memory usage is not highly limited.
This could done e.g. when dealing with high frequency modes.
Another area is usage of graphics processing unit (GPU) for calculation.
\bullet\ already has already experimental GPU pipeline but most constraint types are not 
yet supported. GPU support was one reason for selecting
it to be used in this study

Integration to 3D modelling and animation software products 
should also be solved before we can expect to
see realistic plasticity in main stream games. 
\bullet\ is already integrated into several ones and it was another significant reason for selecting
it to be used in this study.

As already noted in \cite{cg1988}, the modeling of inelastic deformation
remains open for further exploration in the context of computer graphics.

