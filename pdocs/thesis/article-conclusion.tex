\section{Conclusion}

This study presents an approach to account for plastic deformation in 
a velocity based formulation.
In the introduced method, the plastic deformation takes place if the force or moment exceeds the given 
limit, the deformation absorbs energy and the joint breaks if plastic capacity is exceeded. 
The approach presented in this work can be used in the gaming industry to provide realistic 
simulations. 
For gaming purposes the presented method works 
best in scenarios where the connected parts are heavy. This allows a normal 
integration timestep to be used without stability issues. 

This kind of methodology also has considerable potential for the combination of
the structural analysis methods with real-time simulation frameworks.
Handling of significant axial and shear forces and moments at the same time
is one possible area of further studies.

There is also work to be done in the area of performance optimization.
One possible area is the reuse of once calculated values if memory capacity is not limited.
This could done, e.g., when dealing with high frequency modes.
Another area is usage of the graphics processing unit (GPU) for calculation.
\bullet\ already has an experimental GPU pipeline but most constraint types are not 
yet supported. GPU support was one reason for selecting
\bullet\ for use in this study

Integration to 3D modelling and animation software products 
should also be solved before we can expect to
see realistic plasticity in main stream games. 
\bullet\ is already integrated into several ones like Blender 
and it was another significant reason for selecting
\bullet\  to be used in this study.

As already noted in \cite{cg1988}, the modeling of inelastic deformation
remains open for further exploration in the context of computer graphics.

