\section{Introduction}

This work tries to invite structural engineers to take active part in development of more realistic physis engines.
Current computer systems provide enough computing performance and efficient algorithms so collision detection and 
rigid body dynamics can be handled in real time in most scenarios.
Soft body dynamics in which  bodies are deformable is in many cases still too complicated for real time applications.
Finite element method will probably replace other methods by providing general solution at some time but not in near future.
In this work focus is on method where rigid bodies are connected by constraints.



Figure \ref{fig:areas} summarizes branches involved in this work.

\begin{figure}[htb!]
\centering
\begin{tikzpicture}
  \begin{scope}[blend group = soft light]
    \fill[red!30!white]   ( 90:3.2) circle (5);
    \fill[green!30!white] (210:3.2) circle (5);
    \fill[blue!30!white]  (330:3.2) circle (5);
  \end{scope}
  \node at ( 90:5)   [align=left] {solid mechanics\\
structural analysis\\
material plasticity};
  \node at ( 210:4)   [align=left] {mechanics\\
physics engine\\
rigid bodies\\
constraints};
  \node at ( 330:5.5)   [align=left] {computer sciences\\
software engineering\\
extending of bulletphysics\\
verification
};
  \node {\begin{varwidth}{2cm}handling of plasticity in physics\\
engine\end{varwidth}};
\end{tikzpicture}
\caption{Multidisciplinary approach to plasticity in physics engine.}
\label{fig:areas}
\end{figure}

\subsection{Structural analysis}

Structural analysis for advanced structures is nowadays typically done using finite element method and 
analysis time is typically hours. Finite element method is already used in physics engines and usage will come 
more widespread in future as more computing capacity and memory will be available.

In this work all theories in area of structural analysis are old and e.g. 
\cite{timoshenko} is typical reference. Main focus is on elastic and 
linear plastic parts of stress-strain curve and elastic and plastic section modulus.

Typical stress strain curve of ductile steel is shown in \ref{fig:areas}.
Stress-strain curve is not drawn to scale as elastic strain could not be seen as it is typically 0.001 to 0.005.
Straing hardening is taken into account mainly by assuming that plasticity in bending expands.
This can be seen e.g. by bending paperclip. It does not break at low angles but can take few full bends. 

\begin{figure}[htb!]
\centering
\begin{tikzpicture}
\coordinate (Y) at (0.2,4);
\draw[->] (0,0) -- (10,0) node[right] {\large{$\epsilon$}};
\draw[->] (0,0) -- (0,6) node[above] {\large{$\sigma$}};
\draw(0,0) -- (Y) -- (2,4) .. controls (7,6) .. (10,5);
\draw[dashed](0,4) -- (Y);
\node at (-0.2,4) [align=right] {$f_y$};
\end{tikzpicture}
\caption{Stress-strain curve of ductile steel (not to scale).}
\label{fig:areas}
\end{figure}

Difference between elastic and plastic section modulus is shown in \ref{fig:wp}. 

\begin{figure}[htb!]
\centering
\begin{tikzpicture}
\coordinate (S) at (2.5,5);
\draw (0,5) -- (4,5) ;
\draw (0,0) -- (4,0) ;
\draw (2,0) -- (2,5) ;
\draw (1.5,0) -- (S); 
\node[above] at (S) [align=center] {\large{$\sigma<f_y$}};
\end{tikzpicture}
\hspace{1cm}
\begin{tikzpicture}
\coordinate (S) at (3,5);
\draw (0,5) -- (4,5) ;
\draw (0,0) -- (4,0) ;
\draw (2,0) -- (2,5) ;
\draw (1,0) -- (1,2.5) -- (3,2.5) -- (S); 
\node[above] at (S) [align=center] {\large{$\sigma=f_y$}};
\end{tikzpicture}
\caption{Stress distribution under elastic and plastic loads.}
\label{fig:wp}
\end{figure}

If stress is below yield limit, stress and strain are linear within cross section.
If cross section is fully plastic, stress is assumed to be at yield level over whole cross section and 
so plastic section modulus is higher than elastic section modulus.

Elastic part is often ignored in this work as displacements due to to elastic deformation are small and 
related frequencies are high.

E.g. \cite{camp} is good example of university cource providing needed information about plasticity.

\subsection{Physics engine}

Physics engines is active research area on many research and industry areas.
Especially gaming and movie industry are powerful drivers.
Robotics is typical research areas that uses physics engines.
In \lut simulation of working machines has been active research area, e.g. \cite{moisio.thesis}.

Physics engines cannot used in serious structural analysis but using structural analysis in physics engines is not rare.  
E.g. \cite{Obrien:1999:GMA} used finite element model to analyze crack initiation and propagation.

Already \cite{cg1988} used similar principles as in this work to handle plasticity and there are a few further
papers that cite \cite{cg1988}. E.g. following papers provide alternative solutions to similar kind of simulations as 
are handled in this work 
\begin{itemize}
\item \cite{gibson1997survey} presents extensive study on deformable modelling in computer graphics
\item \cite{o2002graphical} provides excellent paper on ductile behaviour of material
\item \cite{muller2004point} presents a method for modeling and animating of 
elastic and plastic bodies in realtime using point based animation
\item \cite{irving2004invertible} introduces algortithm for  finite element simulation of elastoplastic solids with
large deformations in reasonable time 
\end{itemize}


\cite{erleben.thesis} provides extensive introduction to physics engines. 
He uses term Physics-based Animation and does detailed analysis on many areas like
\begin{itemize}
\item preferring fast solutions rather than exact solutions
\item various methods for solving differential equations
\item processing of constraints
\end{itemize}


E.g. \cite{cornell.cs3152} and \cite{cornell.cs5643} are good examples of  university cources
about physics engines. They also provide additional resources on used technologies.

Few physics engines which have public source

\begin{itemize}
\item \bullet - \cite{bullet} which is used e.g. in Blender and Erwin Coumans
 has received Tehnical Academic Award for creation of the Bullet physics library, \cite{erwin.oscar}
\item Box2D - \cite{box2d} which is used e.g. in Angry Birds
\item RigsOfRods - \cite{ror} which uses large amount of rod elements to simulate deformation and breaking of vehicles
\end{itemize}


\subsection{Software engineering}

Main requirements for software implementation can be summarized
\begin{itemize}
\item plastic deformation takes place if force or moment exceeds given limit
\item plastic deformation absorbs energy
\item after predefined limit of displacement is exceeded constraint is deactivated 
\item results should be in same magnitude as theoretical and test results
\item solution can be integrated to existing dynamics engines
\item additional calculation time for each timestep may not exceed few milliseconds 
\end{itemize}

Few decades ago software engineering and structural analysis were highly coupled
and e.g. \cite{bathe} and \cite{cook} contain detailed instructions on
computer implementation. Those books provided good baseline for many computer 
standalone programs created in \lut for for analysis of 
thin-walled structures \cite{agifap,fsm,vtb}.  
Getting and compiling complicated software packages is currently easy. 
Understanding and getting enough knowledge to extend them requires a fair amount of hard work.

Using calculation methods that have initially planned to be hand calculated gives good starting point for 
getting fast processing. Lack of clean boundary conditions will require additional logic or accepting larger errors.


\subsection{Motivation and objectives of the study}

Target of this work is to find out suitable methods for handling of plasticity and verify them 
in commonly used physics engine \bullet. 
Handling of plasticity basically means that involved bodies may absorb kinetic energy and break. 

There are two major areas of applications for these new features.

Possibility to provide simulated demostrations and verifications of new theories or exercises  would be targeted for 
studies of mechanics and strength of material. 
 
Methods presented in this work can be used in gaming industry to provide 
bodies that behave more realistically than currently without significant extra work. 
Efficient simulation algorithms can also help in saving energy. 
Saving even few wats or percent for single user makes big difference as gaming PCs are estimated to have consumed 
75 TWh/year of electricity globally in 2012, \cite{gaming.energy}.

\subsection{Scope}

In this work one solution to handling of plasticity is presented and verified with few examples.

\subsection{Scientific contribution}

This work opens wide area of further work of combining structural analysis and plasticity with physics engines.
Few examples are shown in sections~\ref{sec:results} and possible areas of further work are discussed in section \ref{subsec:futureWork}.

\cleardoublepage
