\section{Introduction}

This work tries to invite structural engineers to take active part in development of more realistic physis engines.
Current computer systems provide enough computing performance and efficient algorithms so collision detection and 
rigid body dynamics can be handled in real time in most scenarios.
Soft body dynamics in which  object are deformable is in many cases still too complicated for real time applications.
Finite element method will probably replace other methods by providing general solution at some time but not in very near future.
In this work focus is on method where rigid bodies are connected by constraints.

\subsection{Physics engines}

Physics engines is active research area on many research and industry areas.
Especially gaming and movie industry are powerful drivers.
Robotics is typical research areas that uses physics engines.
In \lut simulation of working machines has been active research area, e.g. \cite{moisio.thesis}.

\cite{erleben.thesis} provides extensive introduction to physics engines. 
He uses term Physics-based Animation and does detailed analysis on many areas like
\begin{itemize}
\item preferring fast solutions rather than exact solutions
\item various methods for solving differential equations
\item processing of constraints
\end{itemize}

E.g. \cite{cornell.cs3152} and \cite{cornell.cs5643} are good examples of modern university cources
about physics engines. They also provide additional resources on used technologies.

\subsection{Structural analysis}

Physics engines are not commonly used in structural analysis but using structural analysis in physics engines is not rare.  

Structural analysis for advanced structures is nowadays typically done using finite element method and 
analysis time is typically hours. Finite element method is already used in physics engines and usage will come 
more widespread in future as more computing capacity and memory will be available.

E.g. \cite{Obrien:1999:GMA} used finite element model to analyze crack initiation and propagation.

\subsection{Motivation and objectives of the study}

Target of this work is to find out suitable methods for handling of plasticity and verify them 
in commonly used physics engine bulletphysics, \cite{bulletphysics}. 
Handling of plasticity basically means that objects involved may absorb kinetic energy and break. 

There are two major areas of applications for these new features.

Possibility to provide simulated demostrations and verifications of new theories or exercises  would be targeted for 
studies of mechanics and strength of material. Implementation of computer programs could be done as part of computer graphics.
 
Methods presented in this work can be used in gaming industry to provide objects that behave more realistically than currently without significant extra work. 

\subsection{Scope}

In this work one solution to handling of plasticity is presented and verified with few examples.

\subsection{Scientific contribution}

This work opens wide area of further work of combining structural analysis and plasticity with physics engines.
Few examples are shown in sections~\ref{sec:results} and possible areas of further work are discussed in \ref{subsec:futureWork}.

\cleardoublepage
