\section{Discussion}

\subsection{Future work}
\label{subsec:futureWork}

\subsubsection{Implementation using GPU}
\bullet\ can already utilize GPU for many operations. 
Analysis on possible issues in used interfaces should be done before integrations.

\subsubsection{Acceptance to current physics engines}

Required work depends on selected dynamics engine. Availability of these features makes similar 
implementation possible.

\begin{itemize}
\item generic spring or motor constraint is already available 
\item possibility to update constraints after each simulation step
\end{itemize}

Current implementation is done based on \bullet\ but it still needs refactoring and testing before it can be 
accepted to main stream. Interfaces should be reafctored so that handling of additional features like large 
displacements and material nonlinearity can be implemented without further changes.

\subsubsection{Large displacements}
Large displacements are typical if plasticity is involved and
large displacements can also take place due to rigid body motion and should be taken into account.

Update of constraints can be done in  many ways but using callbacks allows clean integration i.e. no changes
to solver core are needed.
In \bullet\ there are already three per step callbacks. 

\begin{itemize}
\item updateActions
\item updateActivationState
\item btInternalTickCallback
\end{itemize}

\demolisher already uses updateActions and that seems 
good way to do also other updates. Special care should be taken to avoid unnecessary updates.

\subsubsection{Material nonlinearity}
Update of constraint properties can be done in callbacks in similar way as for large displacements.
Material definition should be defined so that stress-strain curve can be defined as array of pairs.
This is probably not significant feature for simulations in gaming area but interesting feature for research and teaching area.

\subsection{Object subdivision just in time}
Just in time activation could avoid extra overhead from multiple objects and constraints.
Impacts causing plasticity are usually brief and they happen at distinct times. 
Simulation could allocate processing units for handling these and release them after impact has been processed.

\cleardoublepage
