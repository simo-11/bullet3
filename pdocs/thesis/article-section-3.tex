% !TeX root = article.tex
\section{Numerical examples}

\subsection{Block under tension}
In this section, changes to the constraint formulation are exemplified with some simulation steps
of non constrained, rigid and
elastic-plastic examples using numerical values.

\begin{figure}
\centering
\begin{tikzpicture}
\coordinate (O) at (0,0,0);

% axes
\draw[->] (O) -- (1,0,0) node[anchor=north east]{$x$};
\draw[->] (O) -- (0,1,0) node[anchor=north west]{$y$};
\draw[->] (O) -- (0,0,1) node[anchor=south]{$z$};

% 1,3,1 block, base at y=1.5
\draw (1,1.5,0) -- ++(0,3,0) -- ++(-1,0,0);
\draw (0,1.5,1) -- ++(1,0,0) -- ++(0,3,0) -- ++(-1,0,0) -- ++(0,-3,0);
\draw (1,1.5,0) -- ++(0,0,1);
\draw (0,4.5,0) -- ++(0,0,1);
\draw (1,4.5,0) -- ++(0,0,1);

\draw[-Stealth] (0.5,3,0.5) -- ++(0,1.5,0);
\end{tikzpicture}
\caption{Single body model for plasticity processing.}
\label{fig:tensionModel}
\end{figure}

In the first example, the system under investigation
has only one dynamic rigid body which is a three meters high block of 0.04 \% steel enforced concrete. 
The cross section is one square meter.
A constraint is set so that the connecting frame is at the top of the block. 
In this case, the frame could be anywhere but if multiple bodies are involved, the connecting frame must be defined
so that it reflects the scenario under investigation.
Concrete density is $2000\ \frac{kg}{m^3}$ and steel density is $7800\ \frac{kg}{m^3}$. 
Steel is assumed to handle load in the elastic-plastic case. The yield stress of steel is 200 MPa.
Gravity is $10\ \frac{m}{s^2}$ to keep numbers simple. 
Simulation step is $\frac{1}{60}\ s$ and 10 
iterations are done for a single step.
The body is 1.5 meters above rigid ground. In the unconstrained case the body will hit hit the ground at $t$ = 0.548 s 
($\sqrt{2 \cdot 1.5 / 10}$).

A single simulation step is performed using the following substeps. 
In this work, changes are made only to the constraint setup and the update actions.
\begin{enumerate}
\item Apply gravity to each moveable body. 
This step makes programming easier as otherwise each program should add forces due to gravity in each step. 
In this case, $\vec{f}_{ext}$ in  Equation \ref{eq:eomV2} is added by $\lbrace{0,-60000,0}\rbrace$.
\item Predict unconstrained motion for each body controlled by the physics simulation.
Static and kinematic bodies are not processed in this step.
Prediction is done based on current linear and angular velocities of the body.
\item Predict contacts. In this phase, continuous collision detection is done based on simplified bodies. 
Each rigid body is represented by sphere geometry and contact prediction is done if 
the body moves more than a given threshold value during the simulation step. 
Continuous collision detection is configured for each body. 
A typical scenario that requires continuous collision detection is fast moving bodies that would otherwise go through walls.
\item Perform discrete collision detection. All overlapping bodies are processed and manifoldPoints are created for
each detected contact at the end of the simulation step. 
In the unconstrained case, contact is detected after $t$=0.55 s if 
continuous collision detection is not used and manifold distance gets a negative value (-0.058 m).
\item Calculate simulation islands(groups). Bodies that are near each other based 
on contact prediction or discrete collision detection
 or connected with constraints are grouped in the same group.
\item Solve constraints. Both contact and other constraints are processed in this step. 
 This step is subdivided to setup, iterations and finish phases. 
In teh setup phase constraints are queried for positional errors for calculation of $c$ in
Equation \ref{eq:constraintEquation} and maximum and minimum impulses in Equations \ref{eq:lambdaLow} and 
\ref{eq:lambdaHigh}.
In the finish phase constraint forces are calculated if requested and velocities of bodies are updated.
\item Integrate transforms using velocities modified in the previous step.
\item Update actions (callbacks) are called. 
In the elastic-plastic case, plasticity is summed and the equilibrium
point of the elastic part is updated if the maximum force or moment is exceeded.
\item Activation state of bodies is updated. 
To avoid extra calculation bodies are as default put into sleeping state if linear and angular velocities of
the body are less than
given threshold values (default 0.8 and 1.0) or longer than the set time limit (2.0).
\end{enumerate} 

Equation \ref{eq:constraintEquation} is simplified
to \ref{eq:fixedConstraint} in constrained cases.
\begin{itemize}
\item No rotation takes place. $\omega_1$ and $\omega_2$ are zero.
\item Constraint force mixing can be ignored.
\item Only vertical velocity is handled.
\item The other involved body is rigid and it does not move.
\end{itemize} 

\begin{equation} \label{eq:fixedConstraint}
m v_y = c 
\end{equation}

Equations \ref{eq:lambdaLow} and \ref{eq:lambdaHigh} are not active for fixed cases as
there are no upper or lower limit for forces.
For the elastic-plastic case maximum impulse is set to the product of the yield stress, 
area of steel enforcement and time step (1330).

Method \mbox{btSequentialImpulseConstraintSolver::} \mbox{solveGroupCacheFriendlySetup}
in \cbullet\ is used to get values for the internal variables described in Figure \ref{fig:solVars}. 

\begin{figure}
\begin{description}
\item[velError] is calculated using velocities and external impulses of the connected bodies.  
 \begin{description}
\item[In constraint cases,] the main contibutor is the relative speed of the bodies at the joint point.
\item[In contact cases,] the main contributor is the relative speed of the bodies at the point of contact. 
Bodies are not allowed to penetrate each other. Restitution increases velError.
If contact is not penetrating velError is reduced by $penetration * timeStep$.
\end{description}
\item[posError] is provided by the constraint. It is a significant factor in designing stable constraints.
 \begin{description}
 \item[In fixed cases,] the value is about 12 times the actual position error. Factor 12 is based on the time step (60) 
 and the default value of error reduction parameter (erp), which has a value of 0.2 in this context.
 \item[In elastic plastic cases,]  the value is set to zero if the impulse would be larger the than maximum impulse or
the spring simulation cannot be done in a stable way.
 \item[In contact cases,] the value is zero if there is no penetration. For penetration cases it is 
$\frac{-penetration\, erp}{timeStep}$. In contact cases the default value for erp is 0.8.
 \end{description}
\item[rhs (c)] is calculated by velError jInv + posError jInv
\item[jInv] is calculated using masses and inertias of connected bodies and constraint geometry. 
 \begin{description}
\item[In constraint cases,] it is the mass of the body (6000).
\item[In contact cases,] it varies below the mass of the body.
\end{description}
\item[Impulse] is impulse applied to the body during a timestep.
 \begin{description}
\item[In constraint cases,] it is obtained from the btJointFeedback structure.
\item[In contact cases,] it is obtained by summing the applied impulses from the active manifolds.
\end{description}
\item[erp] Error reduction parameter (0...1) is used to handle numerical issues e.q. body drifting. 
Setting erp to 1 would in theory eliminate error in one step but in practice a value of 0.2 - 0.8 is used in most cases.
\end{description}
\caption{Internal variables used in \cbullet\ in constraint solving.}
\label{fig:solVars}
\end{figure}

Actual values for an unconstrained case without continuous collision detection
are shown in Table \ref{tab:freeBlockValues} and 
location, velocity, posError and velError are depicted in Figure \ref{fig:uc}. 
Penetration is detected at time 0.567 s when
{\it velError} is 5.67 (5.5+0.167) and
{\it posError} is 2.8 (0.058*0.8/0.0167). Impulse is 34000 and contact force is thus about 2 MN (34000/0.0167).
After a few steps the location and position stabilize although internally calculation is needed for each time step
until a body is deactivated.

\begin {table*}
\tbl {Simulation values for the unconstrained case without continuous collision detection. 
Typical values are shown for internal contact values
as the number of contacts and the values at contact points differ.} {
\begin{tabular}{|l| l|l| l|l|l|l|l|}
\hline
{\bf Time} & 
{\bf Location} &
{\it velError} & {\it penetration} & {\it posError} & {\it rhs} &
{\bf Velocity} & 
{\bf Impulse} \\  \hline
0.017 &  -0.003 & & & &  &-0.17 & 0 \\  \hline
0.533 &  -1.467 & & & & & -5.33 & 0 \\  \hline
0.550 &  -1.558 & & & & & -5.5 & 0 \\  \hline
0.567 &  -1.511 & 5.67 &-0.058 &2.8 &  21270 & 0.01 & 34000 \\  \hline
0.583 &  -1.502 & 0.14 &-0.011 & 0.54& 2570  & 0.55 & 420 \\  \hline
0.600 &  -1.496 & -0.38&-0.002 & 0.1  & -1000& 0.38 & 0 \\  \hline
0.617 &  -1.492 &-0.44 & 0.004 & 0     & -1600& 0.22 & 0 \\  \hline
0.717 &  -1.497 &0.004-0.08  &-0.0003-0.001 &0-0.01 & 10-400 & -0.01 & 400 \\  \hline
0.817 &  -1.499 & & & & & -0.08 & 700 \\  \hline
0.917 &  -1.500 & & & & & -0.001 & 1000 \\  \hline
\end {tabular}}
\label{tab:freeBlockValues} 
\end {table*}


\begin{figure}
% Title: glps_renderer figure
% Creator: GL2PS 1.3.8, (C) 1999-2012 C. Geuzaine
% For: Octave
% CreationDate: Tue Oct 11 18:19:22 2016
\begin{pgfpicture}
\pgfsetlinewidth{0.01pt}
\color[rgb]{1.000000,1.000000,1.000000}
\pgfpathmoveto{\pgfpoint{263.105469pt}{383.769531pt}}
\pgflineto{\pgfpoint{263.105469pt}{251.332031pt}}
\pgflineto{\pgfpoint{74.882813pt}{251.332031pt}}
\pgfpathclose
\pgfusepath{fill,stroke}
\pgfpathmoveto{\pgfpoint{74.882813pt}{251.332031pt}}
\pgflineto{\pgfpoint{74.882813pt}{383.769531pt}}
\pgflineto{\pgfpoint{263.105469pt}{383.769531pt}}
\pgfpathclose
\pgfusepath{fill,stroke}
\pgfpathmoveto{\pgfpoint{333.050781pt}{47.523438pt}}
\pgflineto{\pgfpoint{333.050781pt}{179.960938pt}}
\pgflineto{\pgfpoint{521.277344pt}{179.960938pt}}
\pgfpathclose
\pgfusepath{fill,stroke}
\pgfpathmoveto{\pgfpoint{333.050781pt}{251.332031pt}}
\pgflineto{\pgfpoint{333.050781pt}{383.769531pt}}
\pgflineto{\pgfpoint{521.277344pt}{383.769531pt}}
\pgfpathclose
\pgfusepath{fill,stroke}
\pgfpathmoveto{\pgfpoint{521.277344pt}{383.769531pt}}
\pgflineto{\pgfpoint{521.277344pt}{251.332031pt}}
\pgflineto{\pgfpoint{333.050781pt}{251.332031pt}}
\pgfpathclose
\pgfusepath{fill,stroke}
\pgfpathmoveto{\pgfpoint{521.277344pt}{179.960938pt}}
\pgflineto{\pgfpoint{521.277344pt}{47.523438pt}}
\pgflineto{\pgfpoint{333.050781pt}{47.523438pt}}
\pgfpathclose
\pgfusepath{fill,stroke}
\pgfpathmoveto{\pgfpoint{74.882813pt}{47.523438pt}}
\pgflineto{\pgfpoint{74.882813pt}{179.960938pt}}
\pgflineto{\pgfpoint{263.105469pt}{179.960938pt}}
\pgfpathclose
\pgfusepath{fill,stroke}
\pgfpathmoveto{\pgfpoint{263.105469pt}{179.960938pt}}
\pgflineto{\pgfpoint{263.105469pt}{47.523438pt}}
\pgflineto{\pgfpoint{74.882813pt}{47.523438pt}}
\pgfpathclose
\pgfusepath{fill,stroke}
\color[rgb]{0.000000,0.000000,0.000000}
\pgfsetlinewidth{0.500000pt}
\pgfsetdash{{16pt}{0pt}}{0pt}
\pgfpathmoveto{\pgfpoint{112.527344pt}{253.222656pt}}
\pgflineto{\pgfpoint{112.527344pt}{251.332031pt}}
\pgfusepath{stroke}
\pgfpathmoveto{\pgfpoint{112.527344pt}{381.882813pt}}
\pgflineto{\pgfpoint{112.527344pt}{383.769531pt}}
\pgfusepath{stroke}
\pgfpathmoveto{\pgfpoint{150.171875pt}{253.222656pt}}
\pgflineto{\pgfpoint{150.171875pt}{251.332031pt}}
\pgfusepath{stroke}
\pgfpathmoveto{\pgfpoint{150.171875pt}{381.882813pt}}
\pgflineto{\pgfpoint{150.171875pt}{383.769531pt}}
\pgfusepath{stroke}
\pgfpathmoveto{\pgfpoint{187.816406pt}{253.222656pt}}
\pgflineto{\pgfpoint{187.816406pt}{251.332031pt}}
\pgfusepath{stroke}
\pgfpathmoveto{\pgfpoint{187.816406pt}{381.882813pt}}
\pgflineto{\pgfpoint{187.816406pt}{383.769531pt}}
\pgfusepath{stroke}
\pgfpathmoveto{\pgfpoint{225.460938pt}{253.222656pt}}
\pgflineto{\pgfpoint{225.460938pt}{251.332031pt}}
\pgfusepath{stroke}
\pgfpathmoveto{\pgfpoint{225.460938pt}{381.882813pt}}
\pgflineto{\pgfpoint{225.460938pt}{383.769531pt}}
\pgfusepath{stroke}
\pgfpathmoveto{\pgfpoint{263.105469pt}{253.222656pt}}
\pgflineto{\pgfpoint{263.105469pt}{251.332031pt}}
\pgfusepath{stroke}
\pgfpathmoveto{\pgfpoint{263.105469pt}{381.882813pt}}
\pgflineto{\pgfpoint{263.105469pt}{383.769531pt}}
\pgfusepath{stroke}
\pgfsetdash{}{0pt}
\pgfpathmoveto{\pgfpoint{452.386719pt}{66.441406pt}}
\pgflineto{\pgfpoint{414.742188pt}{66.441406pt}}
\pgfusepath{stroke}
\pgfpathmoveto{\pgfpoint{414.742188pt}{66.441406pt}}
\pgflineto{\pgfpoint{377.097656pt}{58.117188pt}}
\pgfusepath{stroke}
\pgfpathmoveto{\pgfpoint{377.097656pt}{58.117188pt}}
\pgflineto{\pgfpoint{370.699219pt}{59.253906pt}}
\pgfusepath{stroke}
\pgfpathmoveto{\pgfpoint{370.699219pt}{59.253906pt}}
\pgflineto{\pgfpoint{364.296875pt}{69.089844pt}}
\pgfusepath{stroke}
\pgfpathmoveto{\pgfpoint{364.296875pt}{69.089844pt}}
\pgflineto{\pgfpoint{358.273438pt}{173.718750pt}}
\pgfusepath{stroke}
\pgfpathmoveto{\pgfpoint{358.273438pt}{173.718750pt}}
\pgflineto{\pgfpoint{351.875000pt}{66.441406pt}}
\pgfusepath{stroke}
\pgfsetdash{{16pt}{0pt}}{0pt}
\pgfpathmoveto{\pgfpoint{76.765625pt}{251.332031pt}}
\pgflineto{\pgfpoint{74.882813pt}{251.332031pt}}
\pgfusepath{stroke}
\pgfpathmoveto{\pgfpoint{261.222656pt}{251.332031pt}}
\pgflineto{\pgfpoint{263.105469pt}{251.332031pt}}
\pgfusepath{stroke}
\pgfpathmoveto{\pgfpoint{76.765625pt}{277.820313pt}}
\pgflineto{\pgfpoint{74.882813pt}{277.820313pt}}
\pgfusepath{stroke}
\pgfpathmoveto{\pgfpoint{261.222656pt}{277.820313pt}}
\pgflineto{\pgfpoint{263.105469pt}{277.820313pt}}
\pgfusepath{stroke}
\pgfpathmoveto{\pgfpoint{76.765625pt}{304.308594pt}}
\pgflineto{\pgfpoint{74.882813pt}{304.308594pt}}
\pgfusepath{stroke}
\pgfpathmoveto{\pgfpoint{261.222656pt}{304.308594pt}}
\pgflineto{\pgfpoint{263.105469pt}{304.308594pt}}
\pgfusepath{stroke}
\pgfpathmoveto{\pgfpoint{76.765625pt}{330.796875pt}}
\pgflineto{\pgfpoint{74.882813pt}{330.796875pt}}
\pgfusepath{stroke}
\pgfpathmoveto{\pgfpoint{261.222656pt}{330.796875pt}}
\pgflineto{\pgfpoint{263.105469pt}{330.796875pt}}
\pgfusepath{stroke}
\pgfpathmoveto{\pgfpoint{76.765625pt}{357.281250pt}}
\pgflineto{\pgfpoint{74.882813pt}{357.281250pt}}
\pgfusepath{stroke}
\pgfpathmoveto{\pgfpoint{261.222656pt}{357.281250pt}}
\pgflineto{\pgfpoint{263.105469pt}{357.281250pt}}
\pgfusepath{stroke}
\pgfpathmoveto{\pgfpoint{76.765625pt}{383.769531pt}}
\pgflineto{\pgfpoint{74.882813pt}{383.769531pt}}
\pgfusepath{stroke}
\pgfpathmoveto{\pgfpoint{261.222656pt}{383.769531pt}}
\pgflineto{\pgfpoint{263.105469pt}{383.769531pt}}
\pgfusepath{stroke}
\pgfsetdash{}{0pt}
\pgfpathmoveto{\pgfpoint{351.875000pt}{66.441406pt}}
\pgflineto{\pgfpoint{345.476563pt}{66.441406pt}}
\pgfusepath{stroke}
\pgfsetdash{{16pt}{0pt}}{0pt}
\pgfpathmoveto{\pgfpoint{519.390625pt}{179.960938pt}}
\pgflineto{\pgfpoint{521.277344pt}{179.960938pt}}
\pgfusepath{stroke}
\pgfpathmoveto{\pgfpoint{334.937500pt}{179.960938pt}}
\pgflineto{\pgfpoint{333.050781pt}{179.960938pt}}
\pgfusepath{stroke}
\pgfpathmoveto{\pgfpoint{519.390625pt}{161.039063pt}}
\pgflineto{\pgfpoint{521.277344pt}{161.039063pt}}
\pgfusepath{stroke}
\pgfpathmoveto{\pgfpoint{334.937500pt}{161.039063pt}}
\pgflineto{\pgfpoint{333.050781pt}{161.039063pt}}
\pgfusepath{stroke}
\pgfpathmoveto{\pgfpoint{519.390625pt}{142.121094pt}}
\pgflineto{\pgfpoint{521.277344pt}{142.121094pt}}
\pgfusepath{stroke}
\pgfpathmoveto{\pgfpoint{334.937500pt}{142.121094pt}}
\pgflineto{\pgfpoint{333.050781pt}{142.121094pt}}
\pgfusepath{stroke}
\pgfsetdash{}{0pt}
\pgfpathmoveto{\pgfpoint{93.707031pt}{253.980469pt}}
\pgflineto{\pgfpoint{87.304688pt}{374.500000pt}}
\pgfusepath{stroke}
\pgfpathmoveto{\pgfpoint{100.105469pt}{316.226563pt}}
\pgflineto{\pgfpoint{93.707031pt}{253.980469pt}}
\pgfusepath{stroke}
\pgfpathmoveto{\pgfpoint{106.128906pt}{328.144531pt}}
\pgflineto{\pgfpoint{100.105469pt}{316.226563pt}}
\pgfusepath{stroke}
\pgfpathmoveto{\pgfpoint{112.527344pt}{336.093750pt}}
\pgflineto{\pgfpoint{106.128906pt}{328.144531pt}}
\pgfusepath{stroke}
\pgfpathmoveto{\pgfpoint{118.925781pt}{341.390625pt}}
\pgflineto{\pgfpoint{112.527344pt}{336.093750pt}}
\pgfusepath{stroke}
\pgfpathmoveto{\pgfpoint{156.570313pt}{334.769531pt}}
\pgflineto{\pgfpoint{118.925781pt}{341.390625pt}}
\pgfusepath{stroke}
\pgfpathmoveto{\pgfpoint{194.218750pt}{332.121094pt}}
\pgflineto{\pgfpoint{156.570313pt}{334.769531pt}}
\pgfusepath{stroke}
\pgfpathmoveto{\pgfpoint{231.863281pt}{330.796875pt}}
\pgflineto{\pgfpoint{194.218750pt}{332.121094pt}}
\pgfusepath{stroke}
\pgfsetdash{{16pt}{0pt}}{0pt}
\pgfpathmoveto{\pgfpoint{74.882813pt}{381.882813pt}}
\pgflineto{\pgfpoint{74.882813pt}{383.769531pt}}
\pgfusepath{stroke}
\pgfpathmoveto{\pgfpoint{74.882813pt}{253.222656pt}}
\pgflineto{\pgfpoint{74.882813pt}{251.332031pt}}
\pgfusepath{stroke}
\pgfpathmoveto{\pgfpoint{263.105469pt}{47.523438pt}}
\pgflineto{\pgfpoint{74.882813pt}{47.523438pt}}
\pgfusepath{stroke}
\pgfpathmoveto{\pgfpoint{263.105469pt}{179.960938pt}}
\pgflineto{\pgfpoint{74.882813pt}{179.960938pt}}
\pgfusepath{stroke}
\pgfpathmoveto{\pgfpoint{74.882813pt}{179.960938pt}}
\pgflineto{\pgfpoint{74.882813pt}{47.523438pt}}
\pgfusepath{stroke}
\pgfpathmoveto{\pgfpoint{263.105469pt}{179.960938pt}}
\pgflineto{\pgfpoint{263.105469pt}{47.523438pt}}
\pgfusepath{stroke}
\pgfpathmoveto{\pgfpoint{74.882813pt}{49.410156pt}}
\pgflineto{\pgfpoint{74.882813pt}{47.523438pt}}
\pgfusepath{stroke}
\pgfpathmoveto{\pgfpoint{74.882813pt}{178.070313pt}}
\pgflineto{\pgfpoint{74.882813pt}{179.960938pt}}
\pgfusepath{stroke}
\pgfpathmoveto{\pgfpoint{112.527344pt}{49.410156pt}}
\pgflineto{\pgfpoint{112.527344pt}{47.523438pt}}
\pgfusepath{stroke}
\pgfpathmoveto{\pgfpoint{112.527344pt}{178.070313pt}}
\pgflineto{\pgfpoint{112.527344pt}{179.960938pt}}
\pgfusepath{stroke}
\pgfpathmoveto{\pgfpoint{150.171875pt}{49.410156pt}}
\pgflineto{\pgfpoint{150.171875pt}{47.523438pt}}
\pgfusepath{stroke}
\pgfpathmoveto{\pgfpoint{150.171875pt}{178.070313pt}}
\pgflineto{\pgfpoint{150.171875pt}{179.960938pt}}
\pgfusepath{stroke}
\pgfpathmoveto{\pgfpoint{187.816406pt}{49.410156pt}}
\pgflineto{\pgfpoint{187.816406pt}{47.523438pt}}
\pgfusepath{stroke}
\pgfpathmoveto{\pgfpoint{187.816406pt}{178.070313pt}}
\pgflineto{\pgfpoint{187.816406pt}{179.960938pt}}
\pgfusepath{stroke}
\pgfpathmoveto{\pgfpoint{225.460938pt}{49.410156pt}}
\pgflineto{\pgfpoint{225.460938pt}{47.523438pt}}
\pgfusepath{stroke}
\pgfpathmoveto{\pgfpoint{225.460938pt}{178.070313pt}}
\pgflineto{\pgfpoint{225.460938pt}{179.960938pt}}
\pgfusepath{stroke}
\pgfpathmoveto{\pgfpoint{263.105469pt}{49.410156pt}}
\pgflineto{\pgfpoint{263.105469pt}{47.523438pt}}
\pgfusepath{stroke}
\pgfpathmoveto{\pgfpoint{263.105469pt}{178.070313pt}}
\pgflineto{\pgfpoint{263.105469pt}{179.960938pt}}
\pgfusepath{stroke}
\pgfpathmoveto{\pgfpoint{519.390625pt}{123.203125pt}}
\pgflineto{\pgfpoint{521.277344pt}{123.203125pt}}
\pgfusepath{stroke}
\pgfpathmoveto{\pgfpoint{334.937500pt}{123.203125pt}}
\pgflineto{\pgfpoint{333.050781pt}{123.203125pt}}
\pgfusepath{stroke}
\pgfpathmoveto{\pgfpoint{519.390625pt}{104.281250pt}}
\pgflineto{\pgfpoint{521.277344pt}{104.281250pt}}
\pgfusepath{stroke}
\pgfpathmoveto{\pgfpoint{334.937500pt}{104.281250pt}}
\pgflineto{\pgfpoint{333.050781pt}{104.281250pt}}
\pgfusepath{stroke}
\pgfpathmoveto{\pgfpoint{519.390625pt}{85.363281pt}}
\pgflineto{\pgfpoint{521.277344pt}{85.363281pt}}
\pgfusepath{stroke}
\pgfpathmoveto{\pgfpoint{334.937500pt}{85.363281pt}}
\pgflineto{\pgfpoint{333.050781pt}{85.363281pt}}
\pgfusepath{stroke}
\pgfpathmoveto{\pgfpoint{76.765625pt}{47.523438pt}}
\pgflineto{\pgfpoint{74.882813pt}{47.523438pt}}
\pgfusepath{stroke}
\pgfpathmoveto{\pgfpoint{261.222656pt}{47.523438pt}}
\pgflineto{\pgfpoint{263.105469pt}{47.523438pt}}
\pgfusepath{stroke}
\pgfpathmoveto{\pgfpoint{263.105469pt}{383.769531pt}}
\pgflineto{\pgfpoint{263.105469pt}{251.332031pt}}
\pgfusepath{stroke}
\pgfpathmoveto{\pgfpoint{261.222656pt}{66.441406pt}}
\pgflineto{\pgfpoint{263.105469pt}{66.441406pt}}
\pgfusepath{stroke}
\pgfpathmoveto{\pgfpoint{76.765625pt}{85.363281pt}}
\pgflineto{\pgfpoint{74.882813pt}{85.363281pt}}
\pgfusepath{stroke}
\pgfpathmoveto{\pgfpoint{261.222656pt}{85.363281pt}}
\pgflineto{\pgfpoint{263.105469pt}{85.363281pt}}
\pgfusepath{stroke}
\pgfpathmoveto{\pgfpoint{76.765625pt}{104.281250pt}}
\pgflineto{\pgfpoint{74.882813pt}{104.281250pt}}
\pgfusepath{stroke}
\pgfpathmoveto{\pgfpoint{261.222656pt}{104.281250pt}}
\pgflineto{\pgfpoint{263.105469pt}{104.281250pt}}
\pgfusepath{stroke}
\pgfpathmoveto{\pgfpoint{76.765625pt}{123.203125pt}}
\pgflineto{\pgfpoint{74.882813pt}{123.203125pt}}
\pgfusepath{stroke}
\pgfpathmoveto{\pgfpoint{261.222656pt}{123.203125pt}}
\pgflineto{\pgfpoint{263.105469pt}{123.203125pt}}
\pgfusepath{stroke}
\pgfpathmoveto{\pgfpoint{76.765625pt}{142.121094pt}}
\pgflineto{\pgfpoint{74.882813pt}{142.121094pt}}
\pgfusepath{stroke}
\pgfpathmoveto{\pgfpoint{261.222656pt}{142.121094pt}}
\pgflineto{\pgfpoint{263.105469pt}{142.121094pt}}
\pgfusepath{stroke}
\pgfpathmoveto{\pgfpoint{76.765625pt}{161.039063pt}}
\pgflineto{\pgfpoint{74.882813pt}{161.039063pt}}
\pgfusepath{stroke}
\pgfpathmoveto{\pgfpoint{261.222656pt}{161.039063pt}}
\pgflineto{\pgfpoint{263.105469pt}{161.039063pt}}
\pgfusepath{stroke}
\pgfpathmoveto{\pgfpoint{76.765625pt}{179.960938pt}}
\pgflineto{\pgfpoint{74.882813pt}{179.960938pt}}
\pgfusepath{stroke}
\pgfpathmoveto{\pgfpoint{261.222656pt}{179.960938pt}}
\pgflineto{\pgfpoint{263.105469pt}{179.960938pt}}
\pgfusepath{stroke}
\pgfpathmoveto{\pgfpoint{519.390625pt}{66.441406pt}}
\pgflineto{\pgfpoint{521.277344pt}{66.441406pt}}
\pgfusepath{stroke}
\pgfpathmoveto{\pgfpoint{334.937500pt}{66.441406pt}}
\pgflineto{\pgfpoint{333.050781pt}{66.441406pt}}
\pgfusepath{stroke}
\pgfpathmoveto{\pgfpoint{519.390625pt}{47.523438pt}}
\pgflineto{\pgfpoint{521.277344pt}{47.523438pt}}
\pgfusepath{stroke}
\pgfpathmoveto{\pgfpoint{334.937500pt}{47.523438pt}}
\pgflineto{\pgfpoint{333.050781pt}{47.523438pt}}
\pgfusepath{stroke}
\pgfpathmoveto{\pgfpoint{521.277344pt}{178.070313pt}}
\pgflineto{\pgfpoint{521.277344pt}{179.960938pt}}
\pgfusepath{stroke}
\pgfpathmoveto{\pgfpoint{521.277344pt}{49.410156pt}}
\pgflineto{\pgfpoint{521.277344pt}{47.523438pt}}
\pgfusepath{stroke}
\pgfpathmoveto{\pgfpoint{483.632813pt}{178.070313pt}}
\pgflineto{\pgfpoint{483.632813pt}{179.960938pt}}
\pgfusepath{stroke}
\pgfpathmoveto{\pgfpoint{483.632813pt}{49.410156pt}}
\pgflineto{\pgfpoint{483.632813pt}{47.523438pt}}
\pgfusepath{stroke}
\pgfpathmoveto{\pgfpoint{445.988281pt}{178.070313pt}}
\pgflineto{\pgfpoint{445.988281pt}{179.960938pt}}
\pgfusepath{stroke}
\pgfsetdash{}{0pt}
\pgfpathmoveto{\pgfpoint{93.707031pt}{66.441406pt}}
\pgflineto{\pgfpoint{87.304688pt}{66.441406pt}}
\pgfusepath{stroke}
\pgfpathmoveto{\pgfpoint{100.105469pt}{172.394531pt}}
\pgflineto{\pgfpoint{93.707031pt}{66.441406pt}}
\pgfusepath{stroke}
\pgfpathmoveto{\pgfpoint{106.128906pt}{86.875000pt}}
\pgflineto{\pgfpoint{100.105469pt}{172.394531pt}}
\pgfusepath{stroke}
\pgfpathmoveto{\pgfpoint{112.527344pt}{70.226563pt}}
\pgflineto{\pgfpoint{106.128906pt}{86.875000pt}}
\pgfusepath{stroke}
\pgfpathmoveto{\pgfpoint{118.925781pt}{66.441406pt}}
\pgflineto{\pgfpoint{112.527344pt}{70.226563pt}}
\pgfusepath{stroke}
\pgfpathmoveto{\pgfpoint{156.570313pt}{66.062500pt}}
\pgflineto{\pgfpoint{118.925781pt}{66.441406pt}}
\pgfusepath{stroke}
\pgfpathmoveto{\pgfpoint{194.218750pt}{66.441406pt}}
\pgflineto{\pgfpoint{156.570313pt}{66.062500pt}}
\pgfusepath{stroke}
\pgfpathmoveto{\pgfpoint{231.863281pt}{66.441406pt}}
\pgflineto{\pgfpoint{194.218750pt}{66.441406pt}}
\pgfusepath{stroke}
\pgfsetdash{{16pt}{0pt}}{0pt}
\pgfpathmoveto{\pgfpoint{74.882813pt}{383.769531pt}}
\pgflineto{\pgfpoint{74.882813pt}{251.332031pt}}
\pgfusepath{stroke}
\pgfpathmoveto{\pgfpoint{263.105469pt}{383.769531pt}}
\pgflineto{\pgfpoint{74.882813pt}{383.769531pt}}
\pgfusepath{stroke}
\pgfpathmoveto{\pgfpoint{521.277344pt}{251.332031pt}}
\pgflineto{\pgfpoint{333.050781pt}{251.332031pt}}
\pgfusepath{stroke}
\pgfpathmoveto{\pgfpoint{521.277344pt}{383.769531pt}}
\pgflineto{\pgfpoint{333.050781pt}{383.769531pt}}
\pgfusepath{stroke}
\pgfpathmoveto{\pgfpoint{333.050781pt}{383.769531pt}}
\pgflineto{\pgfpoint{333.050781pt}{251.332031pt}}
\pgfusepath{stroke}
\pgfpathmoveto{\pgfpoint{521.277344pt}{383.769531pt}}
\pgflineto{\pgfpoint{521.277344pt}{251.332031pt}}
\pgfusepath{stroke}
\pgfpathmoveto{\pgfpoint{333.050781pt}{253.222656pt}}
\pgflineto{\pgfpoint{333.050781pt}{251.332031pt}}
\pgfusepath{stroke}
\pgfpathmoveto{\pgfpoint{333.050781pt}{381.882813pt}}
\pgflineto{\pgfpoint{333.050781pt}{383.769531pt}}
\pgfusepath{stroke}
\pgfpathmoveto{\pgfpoint{370.699219pt}{253.222656pt}}
\pgflineto{\pgfpoint{370.699219pt}{251.332031pt}}
\pgfusepath{stroke}
\pgfpathmoveto{\pgfpoint{370.699219pt}{381.882813pt}}
\pgflineto{\pgfpoint{370.699219pt}{383.769531pt}}
\pgfusepath{stroke}
\pgfpathmoveto{\pgfpoint{408.343750pt}{253.222656pt}}
\pgflineto{\pgfpoint{408.343750pt}{251.332031pt}}
\pgfusepath{stroke}
\pgfpathmoveto{\pgfpoint{408.343750pt}{381.882813pt}}
\pgflineto{\pgfpoint{408.343750pt}{383.769531pt}}
\pgfusepath{stroke}
\pgfpathmoveto{\pgfpoint{445.988281pt}{253.222656pt}}
\pgflineto{\pgfpoint{445.988281pt}{251.332031pt}}
\pgfusepath{stroke}
\pgfpathmoveto{\pgfpoint{445.988281pt}{381.882813pt}}
\pgflineto{\pgfpoint{445.988281pt}{383.769531pt}}
\pgfusepath{stroke}
\pgfpathmoveto{\pgfpoint{483.632813pt}{253.222656pt}}
\pgflineto{\pgfpoint{483.632813pt}{251.332031pt}}
\pgfusepath{stroke}
\pgfpathmoveto{\pgfpoint{483.632813pt}{381.882813pt}}
\pgflineto{\pgfpoint{483.632813pt}{383.769531pt}}
\pgfusepath{stroke}
\pgfpathmoveto{\pgfpoint{521.277344pt}{253.222656pt}}
\pgflineto{\pgfpoint{521.277344pt}{251.332031pt}}
\pgfusepath{stroke}
\pgfpathmoveto{\pgfpoint{521.277344pt}{381.882813pt}}
\pgflineto{\pgfpoint{521.277344pt}{383.769531pt}}
\pgfusepath{stroke}
\pgfpathmoveto{\pgfpoint{445.988281pt}{49.410156pt}}
\pgflineto{\pgfpoint{445.988281pt}{47.523438pt}}
\pgfusepath{stroke}
\pgfpathmoveto{\pgfpoint{408.343750pt}{178.070313pt}}
\pgflineto{\pgfpoint{408.343750pt}{179.960938pt}}
\pgfusepath{stroke}
\pgfpathmoveto{\pgfpoint{408.343750pt}{49.410156pt}}
\pgflineto{\pgfpoint{408.343750pt}{47.523438pt}}
\pgfusepath{stroke}
\pgfpathmoveto{\pgfpoint{370.699219pt}{178.070313pt}}
\pgflineto{\pgfpoint{370.699219pt}{179.960938pt}}
\pgfusepath{stroke}
\pgfpathmoveto{\pgfpoint{370.699219pt}{49.410156pt}}
\pgflineto{\pgfpoint{370.699219pt}{47.523438pt}}
\pgfusepath{stroke}
\pgfpathmoveto{\pgfpoint{333.050781pt}{178.070313pt}}
\pgflineto{\pgfpoint{333.050781pt}{179.960938pt}}
\pgfusepath{stroke}
\pgfpathmoveto{\pgfpoint{334.937500pt}{251.332031pt}}
\pgflineto{\pgfpoint{333.050781pt}{251.332031pt}}
\pgfusepath{stroke}
\pgfpathmoveto{\pgfpoint{519.390625pt}{251.332031pt}}
\pgflineto{\pgfpoint{521.277344pt}{251.332031pt}}
\pgfusepath{stroke}
\pgfpathmoveto{\pgfpoint{334.937500pt}{270.253906pt}}
\pgflineto{\pgfpoint{333.050781pt}{270.253906pt}}
\pgfusepath{stroke}
\pgfpathmoveto{\pgfpoint{519.390625pt}{270.253906pt}}
\pgflineto{\pgfpoint{521.277344pt}{270.253906pt}}
\pgfusepath{stroke}
\pgfpathmoveto{\pgfpoint{334.937500pt}{289.171875pt}}
\pgflineto{\pgfpoint{333.050781pt}{289.171875pt}}
\pgfusepath{stroke}
\pgfpathmoveto{\pgfpoint{519.390625pt}{289.171875pt}}
\pgflineto{\pgfpoint{521.277344pt}{289.171875pt}}
\pgfusepath{stroke}
\pgfpathmoveto{\pgfpoint{334.937500pt}{308.089844pt}}
\pgflineto{\pgfpoint{333.050781pt}{308.089844pt}}
\pgfusepath{stroke}
\pgfpathmoveto{\pgfpoint{519.390625pt}{308.089844pt}}
\pgflineto{\pgfpoint{521.277344pt}{308.089844pt}}
\pgfusepath{stroke}
\pgfpathmoveto{\pgfpoint{334.937500pt}{327.011719pt}}
\pgflineto{\pgfpoint{333.050781pt}{327.011719pt}}
\pgfusepath{stroke}
\pgfpathmoveto{\pgfpoint{519.390625pt}{327.011719pt}}
\pgflineto{\pgfpoint{521.277344pt}{327.011719pt}}
\pgfusepath{stroke}
\pgfpathmoveto{\pgfpoint{334.937500pt}{345.929688pt}}
\pgflineto{\pgfpoint{333.050781pt}{345.929688pt}}
\pgfusepath{stroke}
\pgfpathmoveto{\pgfpoint{519.390625pt}{345.929688pt}}
\pgflineto{\pgfpoint{521.277344pt}{345.929688pt}}
\pgfusepath{stroke}
\pgfpathmoveto{\pgfpoint{334.937500pt}{364.851563pt}}
\pgflineto{\pgfpoint{333.050781pt}{364.851563pt}}
\pgfusepath{stroke}
\pgfpathmoveto{\pgfpoint{519.390625pt}{364.851563pt}}
\pgflineto{\pgfpoint{521.277344pt}{364.851563pt}}
\pgfusepath{stroke}
\pgfpathmoveto{\pgfpoint{334.937500pt}{383.769531pt}}
\pgflineto{\pgfpoint{333.050781pt}{383.769531pt}}
\pgfusepath{stroke}
\pgfpathmoveto{\pgfpoint{519.390625pt}{383.769531pt}}
\pgflineto{\pgfpoint{521.277344pt}{383.769531pt}}
\pgfusepath{stroke}
\pgfpathmoveto{\pgfpoint{333.050781pt}{49.410156pt}}
\pgflineto{\pgfpoint{333.050781pt}{47.523438pt}}
\pgfusepath{stroke}
\pgfpathmoveto{\pgfpoint{521.277344pt}{179.960938pt}}
\pgflineto{\pgfpoint{521.277344pt}{47.523438pt}}
\pgfusepath{stroke}
\pgfpathmoveto{\pgfpoint{333.050781pt}{179.960938pt}}
\pgflineto{\pgfpoint{333.050781pt}{47.523438pt}}
\pgfusepath{stroke}
\pgfpathmoveto{\pgfpoint{521.277344pt}{179.960938pt}}
\pgflineto{\pgfpoint{333.050781pt}{179.960938pt}}
\pgfusepath{stroke}
\pgfpathmoveto{\pgfpoint{521.277344pt}{47.523438pt}}
\pgflineto{\pgfpoint{333.050781pt}{47.523438pt}}
\pgfusepath{stroke}
\pgfpathmoveto{\pgfpoint{263.105469pt}{251.332031pt}}
\pgflineto{\pgfpoint{74.882813pt}{251.332031pt}}
\pgfusepath{stroke}
\pgfpathmoveto{\pgfpoint{76.765625pt}{66.441406pt}}
\pgflineto{\pgfpoint{74.882813pt}{66.441406pt}}
\pgfusepath{stroke}
\pgfsetdash{}{0pt}
\pgfpathmoveto{\pgfpoint{490.031250pt}{364.832031pt}}
\pgflineto{\pgfpoint{452.386719pt}{363.335938pt}}
\pgfusepath{stroke}
\pgfpathmoveto{\pgfpoint{452.386719pt}{363.335938pt}}
\pgflineto{\pgfpoint{414.742188pt}{364.660156pt}}
\pgfusepath{stroke}
\pgfpathmoveto{\pgfpoint{351.875000pt}{260.792969pt}}
\pgflineto{\pgfpoint{345.476563pt}{264.007813pt}}
\pgfusepath{stroke}
\pgfpathmoveto{\pgfpoint{358.273438pt}{365.039063pt}}
\pgflineto{\pgfpoint{351.875000pt}{260.792969pt}}
\pgfusepath{stroke}
\pgfpathmoveto{\pgfpoint{364.296875pt}{375.257813pt}}
\pgflineto{\pgfpoint{358.273438pt}{365.039063pt}}
\pgfusepath{stroke}
\pgfpathmoveto{\pgfpoint{370.699219pt}{372.039063pt}}
\pgflineto{\pgfpoint{364.296875pt}{375.257813pt}}
\pgfusepath{stroke}
\pgfpathmoveto{\pgfpoint{377.097656pt}{369.011719pt}}
\pgflineto{\pgfpoint{370.699219pt}{372.039063pt}}
\pgfusepath{stroke}
\pgfpathmoveto{\pgfpoint{414.742188pt}{364.660156pt}}
\pgflineto{\pgfpoint{377.097656pt}{369.011719pt}}
\pgfusepath{stroke}
\pgfpathmoveto{\pgfpoint{490.031250pt}{66.441406pt}}
\pgflineto{\pgfpoint{452.386719pt}{66.441406pt}}
\pgfusepath{stroke}
{
\pgftransformshift{\pgfpoint{328.046875pt}{383.765625pt}}
\pgfnode{rectangle}{east}{\fontsize{10}{0}\selectfont\textcolor[rgb]{0,0,0}{{1}}}{}{\pgfusepath{discard}}}
{
\pgftransformshift{\pgfpoint{328.046875pt}{364.847656pt}}
\pgfnode{rectangle}{east}{\fontsize{10}{0}\selectfont\textcolor[rgb]{0,0,0}{{0}}}{}{\pgfusepath{discard}}}
{
\pgftransformshift{\pgfpoint{328.046875pt}{345.925781pt}}
\pgfnode{rectangle}{east}{\fontsize{10}{0}\selectfont\textcolor[rgb]{0,0,0}{{-1}}}{}{\pgfusepath{discard}}}
{
\pgftransformshift{\pgfpoint{328.046875pt}{327.007813pt}}
\pgfnode{rectangle}{east}{\fontsize{10}{0}\selectfont\textcolor[rgb]{0,0,0}{{-2}}}{}{\pgfusepath{discard}}}
{
\pgftransformshift{\pgfpoint{328.046875pt}{308.085938pt}}
\pgfnode{rectangle}{east}{\fontsize{10}{0}\selectfont\textcolor[rgb]{0,0,0}{{-3}}}{}{\pgfusepath{discard}}}
{
\pgftransformshift{\pgfpoint{328.046875pt}{289.167969pt}}
\pgfnode{rectangle}{east}{\fontsize{10}{0}\selectfont\textcolor[rgb]{0,0,0}{{-4}}}{}{\pgfusepath{discard}}}
{
\pgftransformshift{\pgfpoint{328.046875pt}{270.250000pt}}
\pgfnode{rectangle}{east}{\fontsize{10}{0}\selectfont\textcolor[rgb]{0,0,0}{{-5}}}{}{\pgfusepath{discard}}}
{
\pgftransformshift{\pgfpoint{328.046875pt}{251.328125pt}}
\pgfnode{rectangle}{east}{\fontsize{10}{0}\selectfont\textcolor[rgb]{0,0,0}{{-6}}}{}{\pgfusepath{discard}}}
{
\pgftransformshift{\pgfpoint{521.277344pt}{246.312500pt}}
\pgfnode{rectangle}{north}{\fontsize{10}{0}\selectfont\textcolor[rgb]{0,0,0}{{1}}}{}{\pgfusepath{discard}}}
{
\pgftransformshift{\pgfpoint{483.632813pt}{246.312500pt}}
\pgfnode{rectangle}{north}{\fontsize{10}{0}\selectfont\textcolor[rgb]{0,0,0}{{0.9}}}{}{\pgfusepath{discard}}}
{
\pgftransformshift{\pgfpoint{445.988281pt}{246.312500pt}}
\pgfnode{rectangle}{north}{\fontsize{10}{0}\selectfont\textcolor[rgb]{0,0,0}{{0.8}}}{}{\pgfusepath{discard}}}
{
\pgftransformshift{\pgfpoint{408.343750pt}{246.312500pt}}
\pgfnode{rectangle}{north}{\fontsize{10}{0}\selectfont\textcolor[rgb]{0,0,0}{{0.7}}}{}{\pgfusepath{discard}}}
{
\pgftransformshift{\pgfpoint{370.699219pt}{246.312500pt}}
\pgfnode{rectangle}{north}{\fontsize{10}{0}\selectfont\textcolor[rgb]{0,0,0}{{0.6}}}{}{\pgfusepath{discard}}}
{
\pgftransformshift{\pgfpoint{333.050781pt}{246.312500pt}}
\pgfnode{rectangle}{north}{\fontsize{10}{0}\selectfont\textcolor[rgb]{0,0,0}{{0.5}}}{}{\pgfusepath{discard}}}
{
\pgftransformshift{\pgfpoint{168.996094pt}{189.957031pt}}
\pgfnode{rectangle}{south}{\fontsize{10}{0}\selectfont\textcolor[rgb]{0,0,0}{{posError}}}{}{\pgfusepath{discard}}}
{
\pgftransformshift{\pgfpoint{69.875000pt}{179.957031pt}}
\pgfnode{rectangle}{east}{\fontsize{10}{0}\selectfont\textcolor[rgb]{0,0,0}{{3}}}{}{\pgfusepath{discard}}}
{
\pgftransformshift{\pgfpoint{69.875000pt}{161.035156pt}}
\pgfnode{rectangle}{east}{\fontsize{10}{0}\selectfont\textcolor[rgb]{0,0,0}{{2.5}}}{}{\pgfusepath{discard}}}
{
\pgftransformshift{\pgfpoint{69.875000pt}{142.117188pt}}
\pgfnode{rectangle}{east}{\fontsize{10}{0}\selectfont\textcolor[rgb]{0,0,0}{{2}}}{}{\pgfusepath{discard}}}
{
\pgftransformshift{\pgfpoint{69.875000pt}{123.199219pt}}
\pgfnode{rectangle}{east}{\fontsize{10}{0}\selectfont\textcolor[rgb]{0,0,0}{{1.5}}}{}{\pgfusepath{discard}}}
{
\pgftransformshift{\pgfpoint{333.050781pt}{42.503906pt}}
\pgfnode{rectangle}{north}{\fontsize{10}{0}\selectfont\textcolor[rgb]{0,0,0}{{0.5}}}{}{\pgfusepath{discard}}}
{
\pgftransformshift{\pgfpoint{370.699219pt}{42.503906pt}}
\pgfnode{rectangle}{north}{\fontsize{10}{0}\selectfont\textcolor[rgb]{0,0,0}{{0.6}}}{}{\pgfusepath{discard}}}
{
\pgftransformshift{\pgfpoint{408.343750pt}{42.503906pt}}
\pgfnode{rectangle}{north}{\fontsize{10}{0}\selectfont\textcolor[rgb]{0,0,0}{{0.7}}}{}{\pgfusepath{discard}}}
{
\pgftransformshift{\pgfpoint{445.988281pt}{42.503906pt}}
\pgfnode{rectangle}{north}{\fontsize{10}{0}\selectfont\textcolor[rgb]{0,0,0}{{0.8}}}{}{\pgfusepath{discard}}}
{
\pgftransformshift{\pgfpoint{483.632813pt}{42.503906pt}}
\pgfnode{rectangle}{north}{\fontsize{10}{0}\selectfont\textcolor[rgb]{0,0,0}{{0.9}}}{}{\pgfusepath{discard}}}
{
\pgftransformshift{\pgfpoint{521.277344pt}{42.503906pt}}
\pgfnode{rectangle}{north}{\fontsize{10}{0}\selectfont\textcolor[rgb]{0,0,0}{{1}}}{}{\pgfusepath{discard}}}
{
\pgftransformshift{\pgfpoint{69.875000pt}{104.277344pt}}
\pgfnode{rectangle}{east}{\fontsize{10}{0}\selectfont\textcolor[rgb]{0,0,0}{{1}}}{}{\pgfusepath{discard}}}
{
\pgftransformshift{\pgfpoint{69.875000pt}{85.359375pt}}
\pgfnode{rectangle}{east}{\fontsize{10}{0}\selectfont\textcolor[rgb]{0,0,0}{{0.5}}}{}{\pgfusepath{discard}}}
{
\pgftransformshift{\pgfpoint{69.875000pt}{66.437500pt}}
\pgfnode{rectangle}{east}{\fontsize{10}{0}\selectfont\textcolor[rgb]{0,0,0}{{0}}}{}{\pgfusepath{discard}}}
{
\pgftransformshift{\pgfpoint{69.875000pt}{47.519531pt}}
\pgfnode{rectangle}{east}{\fontsize{10}{0}\selectfont\textcolor[rgb]{0,0,0}{{-0.5}}}{}{\pgfusepath{discard}}}
{
\pgftransformshift{\pgfpoint{263.105469pt}{42.503906pt}}
\pgfnode{rectangle}{north}{\fontsize{10}{0}\selectfont\textcolor[rgb]{0,0,0}{{1}}}{}{\pgfusepath{discard}}}
{
\pgftransformshift{\pgfpoint{225.460938pt}{42.503906pt}}
\pgfnode{rectangle}{north}{\fontsize{10}{0}\selectfont\textcolor[rgb]{0,0,0}{{0.9}}}{}{\pgfusepath{discard}}}
{
\pgftransformshift{\pgfpoint{187.816406pt}{42.503906pt}}
\pgfnode{rectangle}{north}{\fontsize{10}{0}\selectfont\textcolor[rgb]{0,0,0}{{0.8}}}{}{\pgfusepath{discard}}}
{
\pgftransformshift{\pgfpoint{150.171875pt}{42.503906pt}}
\pgfnode{rectangle}{north}{\fontsize{10}{0}\selectfont\textcolor[rgb]{0,0,0}{{0.7}}}{}{\pgfusepath{discard}}}
{
\pgftransformshift{\pgfpoint{112.527344pt}{42.503906pt}}
\pgfnode{rectangle}{north}{\fontsize{10}{0}\selectfont\textcolor[rgb]{0,0,0}{{0.6}}}{}{\pgfusepath{discard}}}
{
\pgftransformshift{\pgfpoint{74.882813pt}{42.503906pt}}
\pgfnode{rectangle}{north}{\fontsize{10}{0}\selectfont\textcolor[rgb]{0,0,0}{{0.5}}}{}{\pgfusepath{discard}}}
{
\pgftransformshift{\pgfpoint{168.996094pt}{393.765625pt}}
\pgfnode{rectangle}{south}{\fontsize{10}{0}\selectfont\textcolor[rgb]{0,0,0}{{Location}}}{}{\pgfusepath{discard}}}
{
\pgftransformshift{\pgfpoint{69.875000pt}{383.765625pt}}
\pgfnode{rectangle}{east}{\fontsize{10}{0}\selectfont\textcolor[rgb]{0,0,0}{{-1.46}}}{}{\pgfusepath{discard}}}
{
\pgftransformshift{\pgfpoint{69.875000pt}{357.277344pt}}
\pgfnode{rectangle}{east}{\fontsize{10}{0}\selectfont\textcolor[rgb]{0,0,0}{{-1.48}}}{}{\pgfusepath{discard}}}
{
\pgftransformshift{\pgfpoint{69.875000pt}{330.792969pt}}
\pgfnode{rectangle}{east}{\fontsize{10}{0}\selectfont\textcolor[rgb]{0,0,0}{{-1.5}}}{}{\pgfusepath{discard}}}
{
\pgftransformshift{\pgfpoint{69.875000pt}{304.304688pt}}
\pgfnode{rectangle}{east}{\fontsize{10}{0}\selectfont\textcolor[rgb]{0,0,0}{{-1.52}}}{}{\pgfusepath{discard}}}
{
\pgftransformshift{\pgfpoint{69.875000pt}{277.816406pt}}
\pgfnode{rectangle}{east}{\fontsize{10}{0}\selectfont\textcolor[rgb]{0,0,0}{{-1.54}}}{}{\pgfusepath{discard}}}
{
\pgftransformshift{\pgfpoint{328.046875pt}{47.519531pt}}
\pgfnode{rectangle}{east}{\fontsize{10}{0}\selectfont\textcolor[rgb]{0,0,0}{{-1}}}{}{\pgfusepath{discard}}}
{
\pgftransformshift{\pgfpoint{328.046875pt}{66.437500pt}}
\pgfnode{rectangle}{east}{\fontsize{10}{0}\selectfont\textcolor[rgb]{0,0,0}{{0}}}{}{\pgfusepath{discard}}}
{
\pgftransformshift{\pgfpoint{328.046875pt}{85.359375pt}}
\pgfnode{rectangle}{east}{\fontsize{10}{0}\selectfont\textcolor[rgb]{0,0,0}{{1}}}{}{\pgfusepath{discard}}}
{
\pgftransformshift{\pgfpoint{328.046875pt}{104.277344pt}}
\pgfnode{rectangle}{east}{\fontsize{10}{0}\selectfont\textcolor[rgb]{0,0,0}{{2}}}{}{\pgfusepath{discard}}}
{
\pgftransformshift{\pgfpoint{328.046875pt}{123.199219pt}}
\pgfnode{rectangle}{east}{\fontsize{10}{0}\selectfont\textcolor[rgb]{0,0,0}{{3}}}{}{\pgfusepath{discard}}}
{
\pgftransformshift{\pgfpoint{328.046875pt}{142.117188pt}}
\pgfnode{rectangle}{east}{\fontsize{10}{0}\selectfont\textcolor[rgb]{0,0,0}{{4}}}{}{\pgfusepath{discard}}}
{
\pgftransformshift{\pgfpoint{328.046875pt}{161.035156pt}}
\pgfnode{rectangle}{east}{\fontsize{10}{0}\selectfont\textcolor[rgb]{0,0,0}{{5}}}{}{\pgfusepath{discard}}}
{
\pgftransformshift{\pgfpoint{328.046875pt}{179.957031pt}}
\pgfnode{rectangle}{east}{\fontsize{10}{0}\selectfont\textcolor[rgb]{0,0,0}{{6}}}{}{\pgfusepath{discard}}}
{
\pgftransformshift{\pgfpoint{427.164063pt}{189.957031pt}}
\pgfnode{rectangle}{south}{\fontsize{10}{0}\selectfont\textcolor[rgb]{0,0,0}{{velError}}}{}{\pgfusepath{discard}}}
{
\pgftransformshift{\pgfpoint{69.875000pt}{251.328125pt}}
\pgfnode{rectangle}{east}{\fontsize{10}{0}\selectfont\textcolor[rgb]{0,0,0}{{-1.56}}}{}{\pgfusepath{discard}}}
{
\pgftransformshift{\pgfpoint{263.105469pt}{246.312500pt}}
\pgfnode{rectangle}{north}{\fontsize{10}{0}\selectfont\textcolor[rgb]{0,0,0}{{1}}}{}{\pgfusepath{discard}}}
{
\pgftransformshift{\pgfpoint{225.460938pt}{246.312500pt}}
\pgfnode{rectangle}{north}{\fontsize{10}{0}\selectfont\textcolor[rgb]{0,0,0}{{0.9}}}{}{\pgfusepath{discard}}}
{
\pgftransformshift{\pgfpoint{187.816406pt}{246.312500pt}}
\pgfnode{rectangle}{north}{\fontsize{10}{0}\selectfont\textcolor[rgb]{0,0,0}{{0.8}}}{}{\pgfusepath{discard}}}
{
\pgftransformshift{\pgfpoint{150.171875pt}{246.312500pt}}
\pgfnode{rectangle}{north}{\fontsize{10}{0}\selectfont\textcolor[rgb]{0,0,0}{{0.7}}}{}{\pgfusepath{discard}}}
{
\pgftransformshift{\pgfpoint{112.527344pt}{246.312500pt}}
\pgfnode{rectangle}{north}{\fontsize{10}{0}\selectfont\textcolor[rgb]{0,0,0}{{0.6}}}{}{\pgfusepath{discard}}}
{
\pgftransformshift{\pgfpoint{74.882813pt}{246.312500pt}}
\pgfnode{rectangle}{north}{\fontsize{10}{0}\selectfont\textcolor[rgb]{0,0,0}{{0.5}}}{}{\pgfusepath{discard}}}
{
\pgftransformshift{\pgfpoint{427.164063pt}{393.765625pt}}
\pgfnode{rectangle}{south}{\fontsize{10}{0}\selectfont\textcolor[rgb]{0,0,0}{{Velocity}}}{}{\pgfusepath{discard}}}
\end{pgfpicture}

\caption{Graph for location, velocity, positional error and velocity error
for the unconstrained case without continuous collision detection.}
\label{fig:uc}
\end{figure}


Actual values for an unconstrained case with continuous collision detection (ccd) using 1.5 
as the radius of the ccd sphere and 0.001 as the ccd motion threshold
are shown in Table \ref{tab:freeBlockValuesWithCcd}. Collision is detected at time 0.550 s when
{\it velError} is  3.5 (5.34+0.167)-0.033/0.0167 and
{\it posError} is  zero all the time as collision is detected before penetration. 
It should be noted that in general the ccd sphere should not extend the actual body as 
premature contacts are created if collision takes place in regions where the ccd sphere extends out of the actual body.

\begin {table}
\tbl {Simulation values for the unconstrained case with continuous collision detection.}{ 
\begin{tabular}{|l| l|l| l|l|l|l|l|}
\hline
{\bf Time} & 
{\bf Location} &
{\it velError} & {\it pene-} & {\it rhs} &
{\bf Velocity} & 
{\bf Impulse} \\ 
 &  & & {\it tration} &  &
 &  \\ 
 \hline
0.017 &  -0.003 & & &  &-0.17 & 0 \\  \hline
0.550 &  -1.500 & 3.5 & 0.033 & 21000& -2 & 21000 \\  \hline
0.567 &  -1.500 & 2.17 & 0 &  8100 & 0.01 & 13000 \\  \hline
0.583 &  -1.500 & 0.15 & 0 & 600  & 0 & 937 \\  \hline
0.600 &  -1.500 & 0.17 & 0 & 600  & 0 & 1040 \\  \hline
\end {tabular}}
\label{tab:freeBlockValuesWithCcd} 
\end {table}

Values for a fixed constraint are shown in Table
\ref{tab:fixedBlockValues}. The constraint is activated in the second step and positional error is corrected
about 20 \% in each step as requested by using an erp value of 0.2.

\begin{table}
\tbl {Constraint parameter values for the fixed constraint} {
\begin{tabular}{|l|l| l| l|l|l|l|}
\hline
{\bf Time} & 
{\bf Location} &
{\it velError} & {\it posError} & {\it rhs} &
{\bf Velocity} & 
{\bf Impulse} \\  \hline
0.017 & -0.0028 & & & 	 & -0.17 & 0 \\  \hline
0.033 & -0.0022 & -0.33 & -0.033 & -2200 & 0.033 & 2200 \\  \hline
0.050 & -0.0018 & -0.13 & -0.027 & -960 & 0.027 & 960 \\  \hline
0.067 & -0.0014 &-0.14 & -0.021 & -970 & 0.021 & 970 \\  \hline
0.35... & 0 &-0.17 & $\approx$ 0 & -1000 &0.0 & 1000 \\  \hline
\end {tabular}}
\label{tab:fixedBlockValues} 
\end {table}

There are currently two alternative six-dof-spring constraint implementations in \cbullet\ and 
in this work elastic-plastic versions of both of them are developed. 

Table  \ref{tab:ep2Parameters} the summarizes most significant parameters for this study.
There are also other parameters, e.g. for controlling joint motors.
An additional equation is created for each additional constraint. 
Enabled springs and motors add one row and
limits add one if the upper and lower limits are the same and two if both the 
upper and the lower limits are defined.

\begin{table}
\tbl {Selected constraint parameters for elastic-plastic constraint 2}{
\begin{tabular}{|l|p{5cm}|}
\hline
{\bf Parameter} & 
{\bf Description} 
\\ \hline
lowerLimit &  
minimum allowed translation or rotation
 \\  \hline
upperLimit &  
maximum allowed translation or rotation
 \\  \hline
springStiffness & elastic spring stiffness
 \\  \hline
enableSpring & defines if spring is active
 \\  \hline
springStiffnessLimited & should elastic behaviour be tuned to avoid instability
 \\  \hline
equilibriumPoint & should elastic behaviour be tuned to avoid instability
 \\  \hline
currentLimit & describes state of constraint \newline
 0: not limited \newline
 3: loLimit=hiLimit \newline 
 4: current value is between loLimit and HiLimit
 \\ \hline
\end {tabular}}
\label{tab:ep2Parameters} 
\end {table}


Values for an elastic-plastic case are shown in Table \ref{tab:epBlockValues}.
The body drops freely during first simulation step and 
gains enough kinetic energy so that higher impulses are needed in the following steps.
This causes plastic strain during the next three steps. 
Positional error is reported as zero at start of simulation because force exceeds
maximum force.  At later phases positional error is not reported to avoid instability.

\begin{table}
\tbl{Constraint parameter values for elastic-plastic constraint}{ 
\begin{tabular}{|l|l| l| l|l|l|l|l|}
\hline
{\bf Time} & 
{\bf Location} &
{\it velError} & 
{\it rhs} &
{\it velocity} & 
{\bf Impulse} & 
{\bf Plastic } \\ 
& & & & & &{\bf strain}\\
\hline
0.017 & -0.0028 &-0.17 & -1000 & -0.17 & 0 & 0 \\  \hline
0.033 & -0.0046 &-0.33 & -2000 & -0.11 &  1330 & 0.001 \\  \hline
0.050 & -0.0056 &-0.28 & -1670 & -0.056 &  1330 & 0.003 \\  \hline
0.067 & -0.0056 &-0.22 & -1340 &  -0.001&  1330 & 0.004\\  \hline
0.083... & -0.0056  & -0.17 & -1000 &  0.0&  1000 & 0.004\\  \hline
\end {tabular}}
\label{tab:epBlockValues} 
\end {table}

Six-dof-spring constraint 2 has an optional feature to avoid unstability by automatically softening the constraint
spring. The feature is activated if the current timeStep is overly large for the spring-mass system simulation.
The feature is triggered based on the equation below 
\begin{equation} \label{eq:frequencyLimited}
\sqrt{ k /  m_{min}} dt > 0.25, 
\end{equation}
where $k$ is the elastic spring stiffness between the bodies,
$m_{min}$ is the smaller of the connected masses or inertias and $dt$ is the integration timestep.

If this feature is active for elastic-plastic constraint 2   
it does not report positional error but  allows full plastic capacity to be used for correcting
the velocity based error instead of the force provided by the spring.
In this scenario the body behaves as depicted in Table \ref{tab:ep2BlockValues}, i.e., it does not move at all.

\begin{table}
\tbl {Constraint parameter values for elastic-plastic constraint 2} {
\begin{tabular}{|l| l| l|l|l|l|}
\hline
{\bf Location} &
{\it velError} &  {\it rhs} &
{\bf Impulse} & 
{\bf Plastic } \\ 
 & & & &{\bf strain}\\
\hline
 0 & -0.17  & -1000  & 1000 & 0 \\  \hline
\end {tabular}}
\label{tab:ep2BlockValues} 
\end {table}

\subsection{Charpy impact test}
In a second numerical example, simulation of a Charpy impact test is
used to benchmark the approach in a more complicated scenario. 
The material is steel and density is 7800 $\frac{kg}{m^{3}}$. Young’s modulus is 200 GPa.
Specimen dimensions are 10x10x55 mm with a 2 mm notch in the middle which is taken into account in the calculations.
Support anvils initially have 40 mm open space between them and their width is 40 mm. 
The hammer is 0.5 m wide and 0.25 m high, thickness is 0.02 m and mass is 19.5 kg.
The hammer has 40 mm draft.
If the specimen bends about 1.9 radians (108 degrees) it can go between the anvils.
The expected energy loss is the product of the plastic moment of section, 
hinge angle needed for the specimen to go through the supports and 
the yield stress of the specimen. For hinge angle of 1.9 radians and yield stress of 400 MPa, expected energy 
loss is 122 J.

\cbullet\ simulations are usually done using a fixed time step of 1/60 s, i.e., 16.67 ms. 
For this case that is too large. 
The default time step was selected to be 5 ms outside impact  and 0.1 ms during impact. 
An automatic time stepping routine changes the timestep so that at angles higher than 0.2 radians 5 ms time step is
used and adjusts it linearly to a selected time step between the angles of 0.2 and 0.05 radians.

Figure \ref{fig:charpy-series} shows three detailed pictures from the simulation.
The first picture shows early phase of simulation. 
In the second picture the hammer has not yet collided with the specimen.
In the last picture the specimen is bending. 
After few more time steps specimen goes between anvils in one peace.
If maximum rotation is set lower than 1.9 radians, joint is deactivated.

\begin{figure}
\includegraphics[height=2.3cm]{figs/article-charpy-1}
\includegraphics[height=2.3cm]{figs/article-charpy-2}
\includegraphics[width=8.65cm]{figs/article-charpy-3}
\caption{Detailed pictures from simulation of a Charpy impact test}
\label{fig:charpy-series}
\end{figure}

Table \ref{tab:ep2ts} shows energy loss for a few timesteps to demonstrate the sensitivity of the solution
using  elastic-plastic constraint 2.

\begin {table}
\tbl {Energy loss for few timesteps}{
\begin{tabular}{| c| c|l|}
\hline
{\bf Timestep[ms]} & {\bf Energy loss [J]} & {\bf Notes} \\ \hline
 0.2 &  170 & specimen penetrates hammer  \\ \hline
 0.1 &  120 & \\ \hline
 0.05 &  150 & \\ \hline
\end {tabular}}
\label{tab:ep2ts} 
\end {table}

Figure \ref{fig:charpy} shows an overview from simulation as the specimen passes between the supports.

\begin{figure}
\centering
\includegraphics[height=4.5cm]{figs/article-charpy}
\caption{Overview of simulation of Charpy impact test}
\label{fig:charpy}
\end{figure}


\subsection{Demolisher}
The third numerical example introduces the demolisher.
The demolisher is a collection of possible scenarios that can be added to games to provide ductile joints between
rigid bodies.  Rigid concrete bodies are connected by steel enforcement.
Concrete density is 2000 $\frac{kg}{m^3}$ and steel density is 7800 $\frac{kg}{m^3}$. 
Steel is assumed to handle load. Yield stress of steel is 200 MPa.
Gravity is 10  $\frac{m}{s^2}$. Simulation step is $\frac{1}{60} $ s and 10 
iterations are done for a single step. Multiple simulation steps are done for each render if needed.

The simulation can be done in real time on modest hardware, e.g. 
on laptop having 2.0 GHz Intel i3 CPU with integrated graphics. 
Figure \ref{fig:demolisher} shows three screenshots from the simulation.

\begin{figure}
\centering
\includegraphics[width=8.65cm]{figs/demolisher-pre}
\includegraphics[height=4.3cm]{figs/demolisher-wip}
\includegraphics[height=4.3cm]{figs/demolisher-done}
\caption{Simulation of Demolisher scenario}
\label{fig:demolisher}
\end{figure}

The vehicle is modelled as a single box btRaycastVehicle with four driving wheels. Wheels are rendered for visual feedback.
In the left picture, the initial state is shown. 
The gate in front is modelled to be so slender that it has visible elastic deflection.
The gate support is modelled as a heavy box.
The bridge has rigid ramps and supports at both ends.
Gravity is applied gradually to avoid collapse of bridge at the start of the simulation.
In the middle picture, the bridge is shown in a state where a car has driven once over the bridge without stopping and
other bodies have taken light damage.
In the right picture, two joints have been broken and the bridge has fallen from teh support due to bending.

