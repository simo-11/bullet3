\section{Processing plasticity}

In this section, changes to constraint formulation are clarified by simple numerical examples.

System has only one dynamic rigid body which is three meters high bocks. Cross section is one square meter.
Density is $2000 kg \over m^3$. Gravity is $10 m\over s^2$. Simulation step is $1\over 60 s$ and 10 
iterations are done for single step.

Single simulation step is done using following substeps. 
In this work, changes are made only to constraing solving and update actions.
\begin{enumerate}
\item Apply gravity
\item Predict unconstrained motion
\item Predict contacts
\item Perform collision detection
\item Calculate simulation islands(groups). Objetcs that are near each other or connected with constraints are grouped in same group.
\item Solve constraints. Both contact and other constraints are processed in this step.
\item Integrate transforms
\item Update actions (callbacks) are called
\item Activation state of bodies is updated. Bodies that are not active
 are typically marked as sleeping and they are not processed.
\end{enumerate} 

Equation \ref{eq:constraintEquation} is simplified
to \ref{eq:fixedConstraint} in this case.
\begin{itemize}
\item No rotation takes place. $\omega_1$ and $\omega_2$ are zeros.
\item Constraint force mixing can be ignored.
\item Only vertical velocity is handled.
\item Other involved body is rigid and it does not move.
\end{itemize} 

\begin{equation} \label{eq:fixedConstraint}
m v_y = c 
\end{equation}

Method btSequentialImpulseConstraintSolver::solveGroupCacheFriendlySetup
in \bullet\ was used to pick up values.

\begin{description}
\item[velError] is calculated using velocities and external impulses of connected objects. 
 In this case, it is dominant contributor for constraint's impulse after initial phase.
\item[posError] is calculated by constraint. It is significant factor in designing stable constraints. 
 In fixed case, value is about 12 times actual position error. Factor 12 is based on time step (60) 
 and default value of error reduction parameter (0.2).
\end{description}

\begin {table}[htb!]
\begin{center}
\begin{tabular}{|c| l| l|}
\hline
{\bf Time} & 
{\bf velError} & {\bf posError} & {\bf rhs} &
{\bf vertical velocity} & 
{\bf constraint impulse [Ns]} \\  \hline
0.017 &  & & & -0.167 & 0 \\  \\hline
0.033 &  0.33 & -0.033 & -2200 & 0.0333 & 2200 \\  \\hline
0.050 &  -0.13 & -0.0266 & -960 & 0.0266 & 960 \\  \\hline
0.067 &  -0.14 & -0.021 & -970 & 0.021 & 970 \\  \\hline
0.35 &  -0.17 & \~0 & -1000 &0.0 & 1000 \\  \\hline
\end {tabular}
\end{center}
\caption {Constraint parameter values for fixed constraint} \label{tab:fixedBlockValues} 
\end {table}



Equations \ref{eq:constraintEquation}, \ref{eq:lambdaLow} and
\ref{eq:lambdaHigh} 
are created for each constraint.

\begin{equation} \label{eq:constraintEquation}
J_1 v_1 + \Omega_1 \omega_1 + J_2 v_2 + \Omega_2 \omega_2 = c + C \lambda
\end{equation}

\begin{equation} \label{eq:lambdaLow}
\lambda \geq l
\end{equation}

\begin{equation} \label{eq:lambdaHigh}
\lambda \leq h
\end{equation}

Main parameters  and corresponding fields in \bullet\  
 are described in table \ref{tab:constraintParameters}.

\begin {table}[htb!]
\begin{center}
\begin{tabular}{|c| l| l|}
\hline
{\bf Parameter} & {\bf Description} & {\bf btConstraintInfo2 pointer}\\  \hline
$J_1, \Omega_1$ & Jacobian & m\_J1linearAxis, m\_J1angularAxis \\ 
$J_2, \Omega_2$ & & m\_J2linearAxis, m\_J2angularAxis \\ \hline
$v$ & linear velocity & \\ \hline
$\omega$ & angular velocity & \\ \hline
$c$        &  right side vector   & m\_constraintError \\ \hline
$C$  & constraint force mixing & cfm \\  \hline
$\lambda$ & constraint force &  \\ \hline
$l$ & low limit for constraint force & m\_lowerLimit \\ \hline
$h$ & high limit for constraint force & m\_upperLimit \\ \hline
\end {tabular}
\end{center}
\caption {Constraint parameters} \label{tab:constraintParameters} 
\end {table}
