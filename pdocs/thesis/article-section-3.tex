% !TeX root = article.tex
\section{Processing plasticity}

In this section, changes to constraint formulation are clarified by doing few simulation steps of non constrained, rigid and
elastic-plastic examples.

\begin{figure}[htb!]
\centering
\begin{tikzpicture}
\coordinate (O) at (0,0,0);

% axes
\draw[->] (O) -- (1,0,0) node[anchor=north east]{$x$};
\draw[->] (O) -- (0,1,0) node[anchor=north west]{$y$};
\draw[->] (O) -- (0,0,1) node[anchor=south]{$z$};

% 1,3,1 block, base at y=1.5
\draw (1,1.5,0) -- ++(0,3,0) -- ++(-1,0,0);
\draw (0,1.5,1) -- ++(1,0,0) -- ++(0,3,0) -- ++(-1,0,0) -- ++(0,-3,0);
\draw (1,1.5,0) -- ++(0,0,1);
\draw (0,4.5,0) -- ++(0,0,1);
\draw (1,4.5,0) -- ++(0,0,1);

\draw[-Stealth] (0.5,3,0.5) -- ++(0,1.5,0);
\end{tikzpicture}
\caption{Single body model for plasticy processing demonstration.}
\label{fig:tensionModel}
\end{figure}

System has only one dynamic rigid body which is three meters high block of 0.04 \% steel enforced concrete. 
Cross section is one square meter.
Constraint is set so that connecting frame is at top of block. 
In this case frame could be anywhere but if multiple bodies are involved, connecting frame must be defined
so that it reflects wanted scenario.
Concrete density is $2000 \frac{kg}{m^3}$ and steel density is $7800 \frac{kg}{m^3}$. 
Steel is assumed to handle load in elastic-plastic case. Yield stress of steel is 200 MPa.
Gravity is $10 \frac{m}{s^2}$. Simulation step is $\frac{1}{60} s$ and 10 
iterations are done for single step.
Body is 1.5 meters above rigid ground. In unconstrained case body will hit hit ground at t=0.548 s 
($\sqrt{2 \cdot 1.5 / 10}$).

Single simulation step is done using following substeps. 
In this work, changes are made only to constraing solving and update actions.
\begin{enumerate}
\item Apply gravity to each non static body. 
This step just makes programming easier as otherwise each program should add forces due to gravity in each step. 
In this case, $\vec{f}_{ext}$ gets added by $\lbrace{0,-60000,0}\rbrace$.
\item Predict unconstrained motion for each non static body. Prediction is done based on current linear and angular velocities of the body.
\item Predict contacts. In this phase, continuous collision detection is done based on simplified objects. 
Each rigid body is represented by sphere and contact prediction is done if body moves more than given threshold value during simulation step. Continuous collision detection is configured for each body and as default it is not active. 
Typical scenario that requires continuous collision detection is fast moving bodies that would otherwise go through walls.
\item Perform discrete collision detection. All overlapping body pairs are processed and manifoldPoints are created for
each detected contact at end of simulation step. 
In unconstrained case, contact is detected after t=0.55 s if continuous collision detection is not used and manifolds distance gets negative value (-0.058 m).
\item Calculate simulation islands(groups). Bodies that are near each other based on contact prediction or discrete collision detection
 or connected with constraints are grouped in same group.
\item Solve constraints. Both contact and other constraints are processed in this step.
\item Integrate transforms using velocities modified in previous step.
\item Update actions (callbacks) are called. 
In elastic-plastic case, plasticity is summed and equilibrium point of elastic part is updated if maximum force or moment is exceeded.
\item Activation state of bodies is updated. 
To avoid extra calculation bodies are as default put to sleeping state if linear and angular velocities of body are less than
given threshold values (default 0.8 and 1.0) more than set limit (2.0).
\end{enumerate} 

Equation \ref{eq:constraintEquation} is simplified
to \ref{eq:fixedConstraint} in constrained cases.
\begin{itemize}
\item No rotation takes place. $\omega_1$ and $\omega_2$ are zeros.
\item Constraint force mixing can be ignored.
\item Only vertical velocity is handled.
\item Other involved body is rigid and it does not move.
\end{itemize} 

\begin{equation} \label{eq:fixedConstraint}
m v_y = c 
\end{equation}

Equations \ref{eq:lambdaLow} and \ref{eq:lambdaHigh} are not active for fixed case.
For elastic-plastic case maximum impulse is set to product of yield stress, area of steel enforcement and time step (1330).

Method btSequentialImpulseConstraintSolver::solveGroupCacheFriendlySetup
in \bullet\ was used to pick up values for internal variables. 

\begin{description}
\item[velError] is calculated using velocities and external impulses of connected bodies.  
 \begin{description}
\item[In constraint cases,] it is dominant contributor for constraint's impulse.
\item[In contact case,] main contributor is bodies relative speed at point of contact. 
Bodies are not allowed to penetrate each other.
\end{description}
\item[posError] is calculated by constraint. It is significant factor in designing stable constraints.
 \begin{description}
 \item[In fixed case,] value is about 12 times actual position error. Factor 12 is based on time step (60) 
 and default value of error reduction parameter (erp) which has value of 0.2 in this context.
 \item[In elastic plastic case,]  value is set to zero if impulse would be larger than maximum impulse or
spring simulation cannot be done in stable way.
 \item[In contact case,] value is zero if there is no penetration. For penetration cases it is 
$\frac{-penetration\, erp}{timeStep}$
 \end{description}
\item[rhs (c)] is calculated by velError jInv + posError jInv
\item[jInv] is calculated using masses and inertias of connected bodies and constraint geometry. 
 \begin{description}
\item[In constraint cases,] it is mass of body (6000).
\item[In contact case,] it varies below mass of body.
\end{description}
\item[Impulse] is impulse applied to body during timestep.
 \begin{description}
\item[In constraint cases,] it is obtained from btJointFeedback structure.
\item[In contact case,] it is obtained by summing applied impulses from active manifolds.
\end{description}
\item[erp] Error reduction parameter (0...1) is used to handle numerical issues e.q. object drifting. 
Setting erp to 1 would in theory eliminate error but in practice value of 0.8 is used in most cases.
\end{description}

Actual values for unconstrained case without continuous collision detection
are shown in \ref{tab:freeBlockValues}. Penetration is detected at time 0.567 s when
{\it velError} is 5.67 (5.5+0.167) and
{\it posError} is 2.8 (0.058*0.8/0.0167). Impulse is 34000 and contact force is thus about 2 MN (34000/0.0167).
After few steps location and position stabilize although internally calculation is needed for each time step
until body is deactivated.

\begin {table}[htb!]
\begin{center}
\begin{tabular}{|l| l|l| l|l|l|l|l|}
\hline
{\bf Time} & 
{\bf Location} &
{\it velError} & {\it penetration} & {\it posError} & {\it rhs} &
{\bf Velocity} & 
{\bf Impulse} \\  \hline
0.017 &  0 & & & &  &-0.17 & 0 \\  \hline
0.550 &  -1.558 & & & & & -5.5 & 0 \\  \hline
0.567 &  -1.511 & 5.67 &-0.058 &2.8 &  21270 & 0.01 & 34000 \\  \hline
0.583 &  -1.502 & 0.14 &-0.011 & 0.54& 2570  & 0.55 & 420 \\  \hline
0.600 &  -1.496 & -0.38&-0.002 & 0.1  & -1000& 0.38 & 0 \\  \hline
0.617 &  -1.492 &-0.44 & 0.004 & 0     & -1600& 0.22 & 0 \\  \hline
0.717 &  -1.497 &0.004-0.08  &-0.0003-0.001 &0-0.01 & 10-400 & -0.01 & 400 \\  \hline
0.817 &  -1.499 & & & & & -0.08 & 700 \\  \hline
0.917 &  -1.500 & & & & & -0.001 & 1000 \\  \hline
\end {tabular}
\end{center}
\caption {Simulation values for unconstrained case. 
For internal contact values typical values are shown
as number of contacts and detailed values differ.} 
\label{tab:freeBlockValues} 
\end {table}

Actual values for unconstrained case with continuous collision detection (ccd) using 1.5 
as radius of ccd sphere and 0.001 as ccd motion threshold
are shown in \ref{tab:freeBlockValuesWithCcd}. Collision is detected at time 0.550 s when
{\it velError} is  3.5 (5.34+0.167)-0.033/0.0167 and
{\it posError} is  0 as collision is detected before penetration. 
It should be noted that in general ccd sphere should not extend actual body as 
premature contacts are created if collision takes place in those regions.

\begin {table}[htb!]
\begin{center}
\begin{tabular}{|l| l|l| l|l|l|l|l|}
\hline
{\bf Time} & 
{\bf Location} &
{\it velError} & {\it penetration} & {\it posError} & {\it rhs} &
{\bf Velocity} & 
{\bf Impulse} \\  \hline
0.017 &  0 & & & &  &-0.17 & 0 \\  \hline
0.550 &  -1.500 & 3.5 & 0.033 & 0 & 21000& -2 & 21000 \\  \hline
0.567 &  -1.500 & 2.17 & 0 & 0  &  8100 & 0.01 & 13000 \\  \hline
0.583 &  -1.500 & 0.15 & 0 & 0 & 600  & 0 & 937 \\  \hline
0.600 &  -1.500 & 0.17 & 0 & 0 & 600  & 0 & 1040 \\  \hline
\end {tabular}
\end{center}
\caption {Simulation values for unconstrained case with continuous collision detection. } 
\label{tab:freeBlockValuesWithCcd} 
\end {table}

Values for fixed constraint are shown in table
\ref{tab:fixedBlockValues}. Constraint is activated in second step and positional error is corrected
about 20 \% in each step as requested by using erp value 0.2.

\begin {table}[htb!]
\begin{center}
\begin{tabular}{|l|l| l| l|l|l|l|}
\hline
{\bf Time} & 
{\bf Location} &
{\it velError} & {\it posError} & {\it rhs} &
{\bf Velocity} & 
{\bf Impulse} \\  \hline
0.017 & -0.0028 &  & & & -0.17 & 0 \\  \hline
0.033 & -0.0022 & 0.33 & -0.033 & -2200 & 0.033 & 2200 \\  \hline
0.050 & -0.0018 & -0.13 & -0.027 & -960 & 0.027 & 960 \\  \hline
0.067 & -0.0014 &-0.14 & -0.021 & -970 & 0.021 & 970 \\  \hline
0.35... & 0 &-0.17 & $\approx$ 0 & -1000 &0.0 & 1000 \\  \hline
\end {tabular}
\end{center}
\caption {Constraint parameter values for fixed constraint} \label{tab:fixedBlockValues} 
\end {table}

There are currently two alternative six-dof-spring constraint implementations in \bullet\ and 
in this work elastic-plastic versions of both of them were developed. 
Values for elastic-plastic case are shown in tables \ref{tab:epBlockValues}.
Body drops freely during first simulation step and 
gains enough kinetic energy so that higher impulses are needed in few following steps.
This causes plastic strain during next three steps.

\begin {table}[htb!]
\begin{center}
\begin{tabular}{|l|l| l| l|l|l|l|l|}
\hline
{\bf Time} & 
{\bf Location} &
{\it velError} & {\it posError} & {\it rhs} &
{\it velocity} & 
{\bf Impulse} & 
{\bf Plastic strain} \\  \hline
0.017 & -0.0028 &        &    &          & -0.17 & 0 & 0 \\  \hline
0.033 & -0.0046 &-0.33 & 0 & -2000 & -0.11 &  1330 & 0.001 \\  \hline
0.050 & -0.0056 &-0.28 & 0 & -1670 & -0.056 &  1330 & 0.003 \\  \hline
0.067 & -0.0056 &-0.22 & 0 & -1340 &  -0.001&  1330 & 0.004\\  \hline
0.083... & -0.0056  & -0.17 & 0 & -1000 &  0.0&  1000 & 0.004\\  \hline
\end {tabular}
\end{center}
\caption {Constraint parameter values for elastic-plastic constraint} \label{tab:epBlockValues} 
\end {table}

Six-dof-spring constraint 2 has feature to avoid unstability by automatically softening constraint
spring if needed.
If this feature is active for elastic-plastic constraint 2.
 \ref{tab:ep2BlockValues}. 


\begin {table}[htb!]
\begin{center}
\begin{tabular}{|l|l| l| l|l|l|l|l|}
\hline
{\bf Time} & 
{\bf Location} &
{\it velError} & {\it posError} & {\it rhs} &
{\bf Velocity} & 
{\bf Impulse} & 
{\bf Plastic strain} \\  \hline
0.017... &  0 & -0.17  & 0 & -1000 & 0       & 1000 & 0 \\  \hline
\end {tabular}
\end{center}
\caption {Constraint parameter values for elastic-plastic constraint 2} \label{tab:ep2BlockValues} 
\end {table}

